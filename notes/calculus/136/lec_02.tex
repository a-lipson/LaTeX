\lecture{2}{2024-04-05}{Test}

$A=(a_{ij})$  $m\times n$ matrix
$A:\R^{n}\longmapsto\R^{m}$ 
$A\vec{x}=\vec{b},\quad \vec{x} \in \R^{n}, \quad \vec{b} \in  \R^{m}$ 

$(A:\vec{b})\underset{\text{reduce}}{\longmapsto}(B:\vec{c})$ echelon form.

1) Solving a system of $m$ equation with $n$ unknowns
\begin{enumerate}[label=(\alph*)]
  \item no solution
    no pivot on the left with non-zero on the rightmost column
  \item unique solution
    want pivot in every column
    reduced echelon form yields identity matrix
  \item infinitely many solutions
    number of pivots is less than the number of columns
    
    column with no pivot or free variable

\end{enumerate}

Just like DEs
$A\vec{x}=\vec{b}$ 

2) Answer questions about the map.
\begin{enumerate}
  \item $\text{ker}A={\vec{0}}$ or $A$ not injective
$\text{ker}\ne{\vec{0}}$ iff there is a free variable.
iff number of pivots equals the number of columns
the number of free variables is the number of vectors which will generate $\text{ker}A$.


  \item A surjective
    $A$ surjecive iff $A\vec{x}=\vec{b}$ has a solution for any $\vec{b}$.
iff there is a pivot in every row.
so, the number of pivots equals the number of rows
(number of pivots is less than or equal to the number of reowa by the definition of a pivot)
  \item A invertible
    $A$ is invertible iff the number of rows equals the number of pivots equals the number of rows.
     $A$ is square and its reduced echelon form is $I$.
\end{enumerate}


3) Worksheet

Let $\vec{v_1},\ldots,\vec{v_n}$ be the columns of $A$.
(images of $\vec{e_1},\ldots,\vec{e_n}$ in $\R^{n}$)

In general, we can view $A\vec{x}=\vec{b}$ as a vector equation
 \[
x_1\vec{v_1}+x_2\vec{v_2}+\cdots+\vec{x_n}\vec{v_n}=\vec{b}
.\] 

\begin{enumerate}
  \item Is $\vec{b}$ in the set generated by ${\vec{v_1},\ldots,\vec{v_n}}$ 
  \item Are $\left\{ \vec{v_1},\ldots,\vec{v_n} \right\} $ linearly independent? (Solving with $\vec{b}=\vec{0}$).
    Yes, if $\vec{x}=\vec{0}$ is the unique solution.
    
    No, if there is at least on other $\vec{x}$. i.e. when there are infinitely many solutions.
    when there is a free variable or the number of pivots is less than the number of columns
  \item Does ${\vec{v_1},\ldots,\vec{v_n}}$ span/generate the target set in $R^{m}$?
    i.e. do we have a solution $\vec{x}$ for any $\vec{b}$?

    i.e. is $A$ surjective?
  \item Is $\vec{v_1},\ldots,\vec{v_n}$ a basis for $R^{m}$?
    the number columns equals the number of pivots equals the number of rows
\end{enumerate}


\proposition{}

Combined Proposition: Summating
Let $S={\vec{v_1},\ldots,\vec{v_n}}$ be a set of vectos in $\R^{n}$.

\begin{enumerate}
  \item $S$ is a linearly independent set iff $A\vec{x}=\vec{0}$ has a unique solution ($A$ has columns  $\vec{v_1},\ldots,\vec{v_n}$)

    iff there are no free variables

    iff the number of pivots equals the number of columns

if $n>m$, $S$ cannot be linearly independent.
$m$ is the number of rows, which is greater or equal to the number of pivots by the definition of a pivot. 

The number of pivots is equal to the number of colutions which is equal to $n$ when $S$ linearly independent

  \item Spans $\R^{m}$ iff $A\vec{x}=\vec{b}$ has a solution for any $b \in \R^{m}$
    (iff we don't get a row of zeros in non-augmented part)

    iff number of pivots equals the number of rows = $m$

    If $n<m$ m then $S$ cannot spand $\R^{m}$.

    the number of pivots is less than or euqla to the number of columns in echelon form.

  \item $S$ is a basis for $\R^{m}$ iff both 1) and 2) hold.

    So, number of columns equals the number of pivots euqals the number of rows.
\end{enumerate}

\proposition{}
Proposition 3.3
If $V$ has a finite basis, ten the number of elements in its basis is unique and is called the dimension of $V$.

\proof{}
Assume ${\vec{v_1},\ldots,\vec{v_n}}$ and ${\vec{w_1},\ldots,\vec{w_n}}$ are two bases for $V$.

Define  $T:\R^{n}\longmapsto V$ 
% need to align these
$\vec{e_i}\longmapsto \vec{v_i}$

$T$ must be an isomorphism:
\[
  T^{-1}:V \longmapsto \R^{n} 
.\] 

$T^{-1}$ must also take a basis of $V$ to a basis of $\R^{n}$.

So ${T^{-1}(\vec{w_1}),\ldots,T^{-1}(\vec{w_n})}$ is a basis of $\R^{n}$. By matrix/pivot arguments above, a basis of $\R^{n}$ has exactly $n$ elements. So $m=n$.

Definition
If $V$ has a finite basis, the number of elements is calle dthe dimension of  $V$. If  $V$ does not have a finite basis, we say it is infinite dimensioned.

$  $
