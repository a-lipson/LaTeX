\documentclass[../index]{subfiles}

\begin{document}

\begin{theorem}
    For $f$ defined on $(c-p)\cup(c+p)$,
    \[\forall\epsilon>0, \exists\delta>0. \quad 0<|x-c|<\delta \implies |f(x)-L|<\epsilon \iff \lim\limits_{x \to c} f(x)=L.\]

    \begin{enumerate}[label=\alph*.]
        \item $\lim\limits_{x \to c} x = c$.
        \item $\lim\limits_{x \to c} |x| = |c|$.
        \item $\lim\limits_{x \to c} k = k$.
        \item $\lim\limits_{x \to c} f(x)=0 \iff \lim\limits_{x \to c} |f(x)|=0$
    \end{enumerate}
\end{theorem}

\begin{theorem}
    \[\lim\limits_{x \to c} f(x)=L \iff \lim\limits_{x \to c^-} f(x) = L \land \lim\limits_{x \to c^+} f(x)=L.\]
\end{theorem}

\begin{theorem}
    $\lim\limits_{x \to c}f(x)=L \land \lim\limits_{x \to c}g(x)=M \implies$,

    \begin{enumerate}[label= (\roman*)]
        \item $\lim\limits_{x \to c} [f(x)+g(x)] = L+M$.
        \item $\lim\limits_{x \to c} [\alpha f(x)] = \alpha L, \quad \alpha\in\mathbb{R}$.
        \item $\lim\limits_{x \to c} [f(x)g(x)] = LM$.
        \item $\lim\limits_{x \to c} \frac{f(x)}{g(x)} = \frac{L}{M}, \quad M\neq0$.
    \end{enumerate}
\end{theorem}

\end{document}