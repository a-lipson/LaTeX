\documentclass[../hw5]{subfiles}

\begin{document}

Let $I=[0,1]$ and $Y_n(t)=t^n$. Show that the sequence $(Y_n)$ is not Cauchy by computing $||Y_n-Y_m||$.

Fix $n,m\in\mathbb{Z}^+$ such that $n>m$.

Let $k$ be the difference between the two indices, $k=n-m$ such that $k>0$.

Then,
\begin{align*}
    ||Y_n-Y_m||&=\underset{t\in I}{\max}|t^n-t^m|\\
    &=\underset{t \in I}{\max}\left[ t^m\big|t^k-1\big| \right].
\end{align*}
But, \[0\leq t^k\leq 1,\quad \forall k \in\mathbb{Z}^+, \forall t \in [0,1] = I.\]

So, \[||Y_n-Y_m||=\underset{t \in I}{\max}\left[ t^m\left( 1-t^k \right) \right].\]

We then wish to find the extrema of this function $t^m\left( 1-t^k \right)$.
\begin{align*}
    0&=\frac{d}{dt}\left[ t^m\left( 1-t^k \right) \right]\\
    &=mt^{m-1}\left( 1-t^k \right)+t^m\left( -kt^{k-1} \right)\\
    &=t^{m-1}\left[ m\left( 1-t^k \right) +t\left( -kt^{k-1} \right)\right]\\
    &=t^{m-1}\left[ m-mt^k-kt^k \right]\\
    &=t^{m-1}\left[ m-(m+k)t^k \right],\\
\end{align*}
which occurs when either,
\[t^{m-1}=0 \implies t=0,\]
or,
\begin{align*}
    m-(m+k)t^k&=0\\
    t^k&=\frac{m}{m+k}\\
    t&={\left( \frac{m}{m+k} \right)}^{1/k}.\\
\end{align*}

We evaluate the function at the endpoints $0,1$ and the extrema $0,{\left( \frac{m}{m+k} \right)}^{1/k}$.
We see that, at $0$ and $1$, the function becomes zero. So, at the last critical point, when $t={\left( \frac{m}{m+k} \right)}^{1/k}$,
\begin{align*}
    &{\left( \frac{m}{m+k} \right)}^{\frac{m}{k}}\left( 1-{\left( \frac{m}{m+k} \right)}^{\frac{k}{k}} \right)\\
    =&{\left( \frac{m}{m+k} \right)}^{\frac{m}{k}}\left( \frac{k}{m+k} \right).\\
\end{align*}

Now, we will consider the value of the function as $k$, the gap between $n$ and $m$, grows large.
\[\lim\limits_{k\to\infty}\left[ {\left( \frac{m}{m+k} \right)}^{\frac{m}{k}}\left( \frac{k}{m+k} \right) \right].\]

We will consider each term in the limit separately and show that they both exist and are finite.

With
\[\lim\limits_{k\to\infty}\frac{k}{m+k},\]
we have a limit of indeterminate form. We apply L'Hôpital's Rule,
\[\lim\limits_{k\to\infty}\frac{1}{1}=1.\]

Next, for
\begin{align*}
    &\lim\limits_{k\to\infty}{\left( \frac{m}{m+k} \right)}^{\frac{m}{k}}\\
    =&{\left[ \lim\limits_{k\to\infty}{\left( \frac{m}{m+k} \right)}^{1/k} \right]}^m,\\
\end{align*}
we will show that the numerator and denominator are both finite and nonzero so that the limit of the quotient is equal to the quotient of the limits.

The numerator, $\lim\limits_{k\to\infty}m^{1/k}=1, \quad m>0$ by S.H.E 11.4.1.

For the denominator, we know that $\lim\limits_{k\to\infty}\frac{\ln{k}}{k}=0$ by S.H.E. 11.4.1.

So,
\begin{align*}
    \lim\limits_{k\to\infty}\frac{\ln{(m+k)}}{k}&=\lim\limits_{k\to\infty}\frac{\ln{m}\ln{k}}{k}\\
    &=\ln{m}\lim\limits_{k\to\infty}\frac{\ln{k}}{k}\\
    &=\ln{m}\cdot 0\\
    &=0.
\end{align*}
Since,
\[{(m+k)}^{1/k}=e^{\ln{\left( {(m+k)}^{1/k} \right)}}=e^{\frac{\ln{(m+k)}}{k}}.\]
Then,
\begin{align*}
    \lim\limits_{k\to\infty}{(m+k)}^{1/k}&=\lim\limits_{k\to\infty}e^{\frac{\ln{(m+k)}}{k}}\\
    &=e^{\lim\limits_{k\to\infty}\frac{\ln{(m+k)}}{k}}\\
    &=e^0\\
    &=1.\\
\end{align*}

Since the limit of the numerator and denominator are both one, then the limit of their quotient is also one.

Since the both limits in the original product exist and equate to one, then,
\[{\left[ \lim\limits_{k\to\infty}{\left( \frac{m}{m+k} \right)}^{1/k} \right]}^m=1^m=1.\]

We had that,
\[||Y_n-Y_m||=\underset{t\in I}{\max}\left[ t^m\left( 1-t^k \right) \right],\]

So, for $k\to\infty$,
\[||Y_n-Y_m||=\underset{t\in I}{\max}\left[ t^m\left( 1-t^k \right) \right]={\left( \frac{m}{m+k} \right)}^{\frac{m}{k}}\left( \frac{k}{m+k} \right)\to1.\]

Since $||Y_n-Y_m||$ goes to one as $n-m$ goes to infinity, then $||Y_n-Y_m||$ cannot be less than any number $\epsilon$ for some large number $N$ with any $n,m\geq N$. So, the sequence $(Y_n)$ does not meet the definition of Cauchy and is therefore not a Cauchy sequence.

\end{document}