\documentclass[../hw9]{subfiles}

\begin{document}

\begin{theorem}
    Every angle inscribed in a semicircle is a right angle
\end{theorem}

\begin{figure*}[ht]
\centering
\begin{tikzpicture}[>=Latex, scale=2]
    \draw[gray] (0,0) -- (4,0) arc (0:180:2) -- cycle;
    
    \coordinate (O) at (2,0); % Origin
    \coordinate (A) at (0,0); % Left corner
    \coordinate (B) at (4,0); % Right corner

    \path (2,0) ++(45:2) coordinate (P);
    
    \draw[->] (A) -- (P);
    \draw[->] (P) -- (B);
    \draw[->] (O) -- (P);
    \draw[->] (O) -- (A);
    \draw[->] (O) -- (B);

    \fill (O) circle (1pt);

    \node[below] at (1,0) {$-\mathbf{a}$};
    \node[below] at (3,0) {$\mathbf{a}$};
    \node[below] at (3,0.9) {$\mathbf{b}$};
    \node[above left] at (2,0.9) {$\mathbf{c}$};
    \node[left] at (3.9,0.8) {$\mathbf{d}$};
\end{tikzpicture}
\end{figure*}

\begin{proof}
    Let $\theta$ be the angle between $\mathbf{c}$ and $\mathbf{d}$.

    We will prove that $\mathbf{c}\cdot\mathbf{d}=0$. This is an equivalent statement to the Theorem.

    Since $\mathbf{c}-\mathbf{a}=\mathbf{b}$, then $\mathbf{c}=\mathbf{a}+\mathbf{b}$.

    We see that $\mathbf{d}=\mathbf{a}-\mathbf{b}$.

    We also notice that $||\mathbf{a}||=||\mathbf{b}||$, which is the radius of the semicircle.

    Then,
    \begin{align*}
        \mathbf{c}\cdot \mathbf{d}&=(\mathbf{a}-\mathbf{b})\cdot(\mathbf{a}+\mathbf{b})\\
        &=\mathbf{a}\cdot\mathbf{a}+\mathbf{a}\cdot\mathbf{b}-\mathbf{a}\cdot\mathbf{b}+\mathbf{b}\cdot\mathbf{b}\\
        &=\mathbf{a}\cdot\mathbf{a}-\mathbf{b}\cdot\mathbf{b}\\
        &={||\mathbf{a}||}^2-{||\mathbf{b}||}^2\\
        &=0.
    \end{align*}

    So $\mathbf{c}\cdot\mathbf{d}=0$.
\end{proof}

\end{document}
