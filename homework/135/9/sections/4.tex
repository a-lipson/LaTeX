\documentclass[../hw9]{subfiles}

\begin{document}

If $\mathbf{a}$, $\mathbf{b}$, and $\mathbf{c}$ are linearly independent, then, for any vector $\mathbf{d}$,
\[\mathbf{d}=\alpha \mathbf{a}+\beta \mathbf{d}+\gamma \mathbf{c}.\]

Since $\mathbf{b},\mathbf{c}$ are linearly independent, they are not parallel. So, $\mathbf{b}\times \mathbf{c}\neq0$.

Additionally, since $\mathbf{a},\mathbf{b},\mathbf{c}$ are linearly independent, then $\mathbf{a}$ is not a linear combination of $\mathbf{b}$ and $\mathbf{c}$. So, $\mathbf{a}$ does not lie in the plane formed by $\mathbf{b}$ and $\mathbf{c}$. 

Thus, by 13.6.4, \[\mathbf{a}\cdot \mathbf{b}\times \mathbf{c}\neq0.\tag{*}\]

To define $\alpha$, we note that either $\mathbf{b}$ or $\mathbf{c}$ crossed with the vector $\mathbf{b}\times \mathbf{c}$ will always be zero since $\mathbf{b}\times \mathbf{c}$ is perpendicular to both $\mathbf{b}$ and $\mathbf{c}$.
Thus, we consider the following,
\begin{align*}
    \mathbf{d}\cdot(\mathbf{b}\times \mathbf{c})&=(\alpha \mathbf{a}+\beta \mathbf{d}+\gamma \mathbf{c})\cdot(\mathbf{b}\times \mathbf{c})\\
    &=\alpha \mathbf{a}\cdot(\mathbf{b} \times \mathbf{c})\\
    \frac{\mathbf{d}\cdot(\mathbf{b} \times \mathbf{c})}{\mathbf{a}\cdot(\mathbf{b} \times \mathbf{c})}&=\alpha.\\
\end{align*}

The denominator, being the triple product of $\mathbf{a}$, $\mathbf{b}$, and $\mathbf{c}$, is not zero by (*), and is therefore non-zero for any arrangement of the vectors in the triple product by 13.4.7.

We repeat the computation above using $\mathbf{a}\times \mathbf{c}$ and $\mathbf{a}\times \mathbf{b}$ to retain only $\mathbf{b}$ or $\mathbf{c}$ respectively.

This yields the following definitions for $\alpha$, $\beta$, and $\gamma$,
\[
    \alpha=\frac{\mathbf{d}\cdot(\mathbf{b} \times \mathbf{c})}{\mathbf{a}\cdot(\mathbf{b} \times \mathbf{c})}, \qquad
    \beta=\frac{\mathbf{d}\cdot(\mathbf{a}\times \mathbf{c})}{\mathbf{b}\cdot(\mathbf{a}\times \mathbf{c})}, \qquad
    \gamma=\frac{\mathbf{d}\cdot(\mathbf{a}\times \mathbf{b})}{\mathbf{c}\cdot(\mathbf{a}\times \mathbf{b})}.
\]

\begin{figure*}[ht]
\centering
\begin{tikzpicture}[>=Latex, vector/.style={->, thick}, dashedVector/.style={->, dashed, thick}]
    % Define vectors a, b, c
    \coordinate (start) at (0,0);
    \coordinate (a) at (2,1);
    \coordinate (b) at ($(a) + (1.5,0.5)$); % b starts where a ends
    \coordinate (c) at ($(b) + (1,-1)$); % c starts where b ends
    
    % Draw vectors a, b, c
    \draw[vector] (start) -- (a) node[midway, above] {$\vec{a}$};
    \draw[vector] (a) -- (b) node[midway, above] {$\vec{b}$};
    \draw[vector] (b) -- (c) node[midway, right] {$\vec{c}$};
    
    % Draw vector d from start to the end of c, dashed
    \draw[dashedVector] (start) -- (c) node[midway, below] {$\vec{d}$};
\end{tikzpicture}
\caption{Linear combination of $\mathbf{a}$, $\mathbf{b}$, and $\mathbf{c}$ to form $\mathbf{d}$.}
\end{figure*}

\end{document}
