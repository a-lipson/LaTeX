\documentclass[../hw9]{subfiles}

\begin{document}

The three points $P_1$, $P_2$, and $Q$ form a plane that intersects the sphere on which they lie to form a circle.

Given that $P_1$ and $P_2$ are antipodal, we can instead consider a semicircle with $P_1$ and $P_2$ at the endpoints of the arc.

With these conditions, we will show that $\overrightarrow{P_1 Q}$ is perpendicular to $\overrightarrow{P_2 Q}$ in problem 5. 

% Let $R$ be the radius of the sphere centered at the origin. 

% Let $\mathbf{Q}$ be the vector to the point $Q$. Let $\mathbf{P}_1$ and $\mathbf{P}_2$ be the vectors to the points $P_1$ and $P_2$ respectively.

% Notice that the the norm of these vectors are all $R$, \[||\mathbf{P}_1||=||\mathbf{P}_2||=||\mathbf{Q}||=R.\]

% Since $\mathbf{P}_1$ and $\mathbf{P}_2$ are antipodal, the angle between the two vectors is $\pi$. Therefore their dot product is,
% \[\mathbf{P}_1\cdot\mathbf{P}_2=||\mathbf{P}_1||||\mathbf{P}_2||\cos{\pi}=-R^2.\]

% Note that the three points $P_1$, $P_2$, and $Q$ line on a plane that bisects the sphere. 

% Let $\theta_1$ be the angle between $\mathbf{P}_1$ and $\mathbf{Q}$; let $\theta_2$ be the angle between $\mathbf{P}_2$ and $\mathbf{Q}$.

% So, \[\theta_1+\theta_2=\pi.\]

% We will show that the angle between the vectors from $\mathbf{Q}$ to $\mathbf{P}_1$, $\mathbf{P}_1-\mathbf{Q}$ and from $\mathbf{Q}$ to $\mathbf{P}_2$, $\mathbf{P}_2-\mathbf{Q}$ is a right angle, and therefore that their dot product is zero.
% \begin{align*}
%     0&=\cos{\theta_1}-\cos{\theta_1}\\
%     &=R^2(\cos{\theta_1}+\cos{(\pi-\theta_1)})\\
%     &=R^2(\cos{\theta_1}+\cos{\theta_2})\\
%     &=||\mathbf{P}_1||||\mathbf{Q}||\cos{\theta_1}+||\mathbf{P}_2||||\mathbf{Q}||\cos{\theta_2}\\
%     &=\mathbf{P}_1\cdot\mathbf{Q}+\mathbf{P}_2\cdot\mathbf{Q}\\
%     &=-R^2-(\mathbf{P}_1\cdot\mathbf{Q}+\mathbf{P}_2\cdot\mathbf{Q})+R^2\\
%     &=\mathbf{P}_1\cdot\mathbf{P}_2-\mathbf{P}_1\cdot\mathbf{Q}-\mathbf{P}_2\cdot\mathbf{Q}+\mathbf{Q}\cdot\mathbf{Q}\\
%     0&=(\mathbf{P}_1-\mathbf{Q})\cdot(\mathbf{P}_2-\mathbf{Q}).\\
% \end{align*}

% Since the dot product of $(\mathbf{P}_1-\mathbf{Q})$ and $(\mathbf{P}_2-\mathbf{Q})$ is zero, then these vectors are orthogonal to each other.

\end{document}
