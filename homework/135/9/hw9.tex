\documentclass{article}

\usepackage[english]{babel}
\usepackage[letterpaper,top=2cm,bottom=2cm,left=3cm,right=3cm,marginparwidth=1.75cm]{geometry}
\usepackage{amsmath, amssymb, amsthm}
\usepackage{comment}
\usepackage{enumitem}
\usepackage{fancyhdr}
\usepackage{graphicx}
\usepackage{hyperref}
\usepackage{ifthen,ifthenx}
\usepackage[skip=15pt,indent=0pt]{parskip}
\usepackage{pgfplots}
\usepackage{tikz}
    \usetikzlibrary{arrows.meta, calc, intersections, pgfplots.fillbetween, angles, quotes}
\usepackage{subfiles}

\pgfplotsset{compat=1.18}
\pgfdeclarelayer{ft} 
\pgfdeclarelayer{bg} 
\pgfsetlayers{bg,main,ft}

\newtheorem*{lemma}{Lemma}
\newtheorem*{proposition}{Proposition}
\newtheorem*{theorem}{Theorem}
\newtheorem*{definition}{Definition}

\def\titleme{Math 135 Homework 9}
\def\authorme{Alexandre Lipson}

\title{\titleme}
\author{\authorme}


\pagestyle{fancy}
\fancyhf{}
\lhead{\today}
\rhead{\authorme}
\chead{\titleme}

\rfoot{\thepage}


\begin{document}
\maketitle

\begin{enumerate}
    \item\subfile{sections/1}
    \item\subfile{sections/2}
    \item\subfile{sections/3}
    \item\subfile{sections/4}
    \item\subfile{sections/5}
\end{enumerate}

%There are two ways to do proofs involving vectors. The messier way involves writing down vectors in their component forms and carrying out the computations. The neater way mimics proofs of the analogous properties with absolute value (which is norm in one-space) and using the dot product and its properties. Try both when you are getting started and eventually try to stop using components.

%Also, keep in mind that when you write a proof, you always start with what you know and end with what you are trying to show. In many cases, this is ”reverse” of how you got the idea or justified the equation/inequality in the first place.

\end{document}
