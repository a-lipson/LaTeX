\documentclass{article}
\begin{document}
\begin{problem}
  Find a closed form for the $n^{\text{th}}$ Fibonacci number $\varphi_n$. The series is defined recursively as $\varphi_{n+2}=\varphi_{n+1}+\varphi_{n}$. 
\end{problem}

First, we will find a matrix $A_{2\times 2}$ such that \[
  \begin{pmatrix} \varphi_{n+2} \\ \varphi_{n+1} \end{pmatrix} =
  A \begin{pmatrix} \varphi_{n+1} \\ \varphi_{n} \end{pmatrix} 
.\] 

This holds for the matrix \[
  A=\begin{pmatrix} 1 & 1 \\ 1 & 0 \\ \end{pmatrix} 
.\] 

We will diagonalize $A$ to find a form for $A^n$.

We compute the eigenvalues of the matrix,
 \begin{align*}
  0&= \text{det}(A-\lambda I) \\
   &= \text{det}\begin{pmatrix} 1-\lambda & 1 \\ 1 & 0-\lambda \end{pmatrix}  \\
   &= (1-\lambda)(-\lambda)-1 \\
   &= \lambda^2-\lambda-1 \\
.\end{align*}

We recognize the solutions to this equation to be the golden ratio $\varphi$ and its conjugate $-\varphi^{-1}$; these are our two eigenvalues $\lambda_1$ and $\lambda_2$. 

So, for diagonalizable $A=SDS^{-1}$, we now have that \[
  D=\begin{pmatrix} \varphi & 0 \\ 0 & -\varphi^{-1} \end{pmatrix} 
.\] 

To determine the isomorphic matrix $S$, we compute the eigenvectors for $A$.

For $\lambda_1=\varphi$, along with the fact that $1-\varphi=-\varphi^{-1}$, $A-\lambda I$ gives, 
\[
  \begin{pmatrix} 1-\varphi & 1 \\ 1 & -\varphi \end{pmatrix} 
  \rightarrow
  \begin{pmatrix} -\varphi^{-1} & 1 \\ \varphi^{-1} & -1 \end{pmatrix} 
  \rightarrow
  \begin{pmatrix} 1 & -\varphi \\ 0 & 0 \end{pmatrix} 
.\] 
which yields the eigenvector $\begin{pmatrix} \varphi \\ 1 \end{pmatrix}$.

Similarly, for $\lambda_2=-\varphi^{-1}$,
\[
  \begin{pmatrix} 1-{(-\varphi)}^{-1} & 1 \\ 1 -{(-\varphi)}^{-1} \end{pmatrix} 
  =
  \begin{pmatrix} \varphi & 1 \\ 1 & \varphi^{-1} \end{pmatrix} 
  \rightarrow
  \begin{pmatrix} \varphi & 1 \\ 0 & 0 \end{pmatrix} 
,\] 
which provides the eigenvector $\begin{pmatrix} -\varphi^{-1} \\ 1 \end{pmatrix}$.

So, \[
  S=\begin{pmatrix} \varphi & -\varphi^{-1} \\ 1 & 1 \end{pmatrix} 
.\] 

Then, for $S^{-1}$, we use the form of the inverse of a $2\times 2$ matrix with the determinant of $S$ to see that
 \[
   S^{-1}=\frac{1}{2\varphi-1}\begin{pmatrix} 1 & \varphi^{-1} \\ -1 & \varphi \end{pmatrix} = \frac{1}{\sqrt{5}}\begin{pmatrix} 1 & \varphi^{-1} \\ -1 & \varphi \end{pmatrix}
.\] 

Then, with $A^n=SD^nS^{-1}$,
\[
  A^n=\begin{pmatrix} \varphi & -\varphi^{-1} \\ 1 & 1 \end{pmatrix}
\begin{pmatrix} \varphi^n & 0 \\ 0 & {(-\varphi^{-1})}^n \end{pmatrix}
\frac{1}{\sqrt{5} }\begin{pmatrix} 1 & \varphi^{-1} \\ -1 & \varphi \end{pmatrix} 
.\] 

Since \[
  \begin{pmatrix} \varphi_{n+1} \\ \varphi_{n} \end{pmatrix} = A^n \begin{pmatrix} \varphi_1 \\ \varphi_0 \end{pmatrix} = A^n \begin{pmatrix} 1 \\ 0 \end{pmatrix},
\] 
then, 
\begin{align*}
  \begin{pmatrix} \varphi_{n+1} \\ \varphi_{n} \end{pmatrix} &= \begin{pmatrix} \varphi & -\varphi^{-1} \\ 1 & 1 \end{pmatrix}
\begin{pmatrix} \varphi^n & 0 \\ 0 & {(-\varphi^{-1})}^n \end{pmatrix}
\frac{1}{\sqrt{5} }\begin{pmatrix} 1 & \varphi^{-1} \\ -1 & \varphi \end{pmatrix} 
\begin{pmatrix} 1 \\ 0 \end{pmatrix} 
\\
&= \frac{1}{\sqrt{5} } \begin{pmatrix} \varphi & -\varphi^{-1} \\ 1 & 1 \end{pmatrix}
\begin{pmatrix} \varphi^n & 0 \\ 0 & {(-\varphi)}^{-n} \end{pmatrix}
\begin{pmatrix} 1 \\ -1 \end{pmatrix} \\
&=  \frac{1}{\sqrt{5} } \begin{pmatrix} \varphi & -\varphi^{-1} \\ 1 & 1 \end{pmatrix}
\begin{pmatrix} \varphi^n \\ -{(-\varphi)}^{-n} \end{pmatrix} \\
&= \frac{1}{\sqrt{5} }\begin{pmatrix} \varphi^{n+1}-{(-\varphi)}^{-(n+1)} \\ \varphi^n-{(-\varphi)}^{-n} \end{pmatrix}.\\
\end{align*}

So, we see that
\[
  \varphi_n = \frac{1}{\sqrt{5} }\left( \varphi^n-{(-\varphi)}^{-n} \right)
.\] 

We will show that $\begin{pmatrix}\frac{\varphi_{n+1}}{\varphi_{n}}\\1\end{pmatrix}$ converges to an eigenvector of $A$. 



\end{document}
