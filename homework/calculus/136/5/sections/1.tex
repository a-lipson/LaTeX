\documentclass{article}

\begin{document}
  Prove that the geometric multiplicity of an eigenvalue cannot exceed its algebraic multiplicity.
\begin{proof}
  Let $\vec{v_1},\ldots,\vec{v_r}$ be a basis of the eigenspace of $A$ for a corresponding eigenvalue $\lambda_0$. So, $r$ is the geometric multiplicity of $\lambda_0$. 

  By 4.1.8, we can complete this to form a basis in $\R^{n}$,
  \[
    A=
    \begin{pmatrix} \hspace{5pt}
      \lambda_0 I_r & * \\ \mathbf{0} & B
  \hspace{5pt} \end{pmatrix} 
  \] 
  where $I_r$ is the $r\times r$ identity matrix.

  By 4.1.7, this characteristic polynomial of this matrix is given by its determinant,
  \[
  \text{det}(A-\lambda I)=\text{det}(\lambda_0 I_g - \lambda I)\text{det}(B-\lambda I)
  .\] 
  This first determinant term will become the $\lambda_0$ roof of degree $r$ and the second determinant will be a polynomial in $\lambda$ which we will call $q(\lambda)$.

  So, \[
    \text{det}(A-\lambda I)={(\lambda_0 - \lambda)}^r q(\lambda)
  .\] 

  Since the determinant does not depend on basis, we will also consider the characteristic equation for the operator $A$ in the standard basis. This will be a polynomial with roots for each eigenvalue  $\lambda_i$ with degree $m_i$ for $i=1,2,\ldots,k$. $m_i$ represents the algebraic multiplicity of the root.
  \[
    P(\lambda)={(\lambda_0-\lambda)}^{m_0}{(\lambda_1-\lambda)}^{m_1}\cdots{(\lambda_k-\lambda)}^{m_k}
  .\] 

  Then, we equate these two characteristic polynomials and divide by the $r^{\text{th}}$ power of the $\lambda_0$ root.
  \[
    {(\lambda_0 - \lambda)}^{m_0-r}{(\lambda_1 - \lambda)}^{m_1}\cdots{(\lambda_k-\lambda)}^{m_k}=q(\lambda)
  .\] 

  But $q$ is a polynomial and not a rational function, so the power of the $\lambda_0$ root must be positive. 

  Hence $m_0 \ge r$; the geometric multiplicity $r$ cannot exceed the algebraic multiplicity $m_0$. 
\end{proof}
\end{document}
