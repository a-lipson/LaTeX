\documentclass[../hw8]{subfiles}

\begin{document}

We will consider the system when there is $y$ centimeters of mercury above equilibrium on one side and $y$ centimeters below equilibrium on the other side. 

So, if we remove the $2y$ column of mercury, the system is at equilibrium and the net force is equal to zero.

Thus the net force in the system, net mass times net acceleration, is given by the weight of the column of mercury of height $2y$. The downward force of the weight of the column of mercury opposes the direction of acceleration of the system.

The acceleration is modeled by the second derivative of $y$.

The volume of this column is $2y$ times the cross sectional area of the cylindrical tube $A=\pi$.

So,
\begin{align*}
    -2yA\rho_L g&= m_L y'' \\
    m_L y''+2yA\rho_L g&= 0 \\
    y'' + \frac{2A\rho_L g}{m_L}y&= 0 \\
    y'' + \frac{2\cdot\pi\cdot13.5\cdot9.8}{500}y &= 0 \\
    y'' + \frac{1323\pi}{2500}y &= 0 \\
\end{align*}

For the characteristic equation,
\begin{align*}
    r^2+\frac{1323\pi}{2500}&=0\\
    r&=\pm\frac{21\sqrt{3\pi}i}{50}.\\
\end{align*}

Let $\omega=\frac{21\sqrt{3\pi}}{50}$.

Then, \[y=c_1\cos{\omega t}+c_2\sin{\omega t}.\]

To find the period $T$, we consider when $\omega (t+T)=\omega t + 2\pi$, which is the period of the sine and cosine functions.

So,
\begin{align*}
    \omega (t+T)&=\omega t + 2\pi\\
    \omega T &= 2\pi\\
    T &= \frac{2\pi}{\omega},\\
    T &= \frac{100\pi}{21\sqrt{3\pi}}\\
    T &= \frac{100\sqrt{\pi}}{21\sqrt{3}}.
\end{align*}

\end{document}
