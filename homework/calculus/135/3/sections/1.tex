\documentclass[../hw3]{subfiles}

\begin{document}

Let the $n^{\text{th}}$ Taylor Polynomial for $f(x)$ at $x=a$ be given by,
\[T_{n,a}(x)=\sum_{k=0}^{n}\frac{f^{(k)}(a)}{k!}{(x-a)}^k.\]

(a) Show that \[f^{(i)}(a)=T^{(i)}_{n,a}(a), \qquad i\in[0,n].\]

% Note that the $n^{\text{th}}$ derivative of $x^k$ is, \[\frac{d^n}{dx^n}x^k=\]

We begin by pulling out the first term of $T_{n,a}(x)$ and taking the $i^{\text{th}}$ derivative of $T_{n,a}(x)$. We can differentiate a sum because of the linearity of the derivative.
\begin{align*}
    \frac{d^i}{dx^i}[T_{n,a}(x)]&=\frac{d^i}{dx^i}\left[ f(a) + \sum_{k=1}^{n}\frac{f^{(k)}(a)}{k^!}{(x-a)}^k \right] \\
    &= \frac{d^i}{dx^i}[f(a)] +\sum_{k=1}^{n} \frac{f^{(k)}(a)}{k!}\left( \frac{d^i}{dx^i}{(x-a)}^k \right)
\end{align*}

We notice that at $x=a$, the sum will reduce to zero given that $x-a=0$. 
So, we are left with the $i^{\text{th}}$ derivative of the series $T$, \[T^{(i)}_{n,a}(a)=f^{(i)}(a),\] which is what we wished to show.

% Notice that all but the first term cancel in $T_{n,a}(x)$ at $x=a$,
% \begin{align*}
%     T_{n,a}(x)&=\sum_{k=0}^{n}\frac{f^{(k)}(a)}{k!}{(x-a)}^k
%     &=f(a)+\sum_{k=1}^{n}\frac{f^{(k)}(a)}{k!}{(x-a)}^k
%     T_{n,a}(a)&=f(a)+\sum_{k=1}^{n}\frac{f^{(k)}(a)}{k!}{(a-a)}^k
%     &=f(a).
% \end{align*}

(b) Show that \[\lim\limits_{x\to a}\frac{f(x)-T_{n,a}(x)}{{(x-a)}^n}=0\] using L'Hôpital's Rule.

First, we note that the $n^{\text{th}}$ derivative of the function ${(x-a)}^n$ is $n!$,
\[\frac{d^n}{dx^n}\left[ {(x-a)}^n \right] = n(n-1)(n-2)\cdots(n-n+1){(x-a)}^{n-n}=n!.\]

We have shown in (a) that the following relationship holds for the $i^{\text{th}}$ derivative up to $i=n$,
\[f^{(i)}(a)=T^{(i)}_{n,a}(a).\]

In order to construct the $n^{\text{th}}$ term of the series $T$, we need $f(x)$ to be at least $n$ times differentiate. So, the $n^{\text{th}}$ derivative of $f$, being differentiable itself, is also continuous. Then, since $T^{(i)}_{n,a}(x)$ is polynomial, it is continuous.

So, we can rewrite the above identity as a limit expression,
\[\lim\limits_{x\to a} f^{(i)}(x)=\lim\limits_{x\to a}T^{(i)}_{n,a}(x),\] for all $i\in[0,n]$.

Next, we notice that the original limit is of indeterminate form. We have already seen that the numberator will equate to zero; the denominator ${(x-a)}^n$ will tend toward zero as $x\to a$.

We then proceed with L'Hôpital's Rule, and differentiate $n$ times, such that the denominator becomes non-zero and the limit expression becomes of determine form.

Thus, \[\lim\limits_{x\to a} \frac{f(x)-T_{n,a}(x)}{{(x-a)}^n} \stackrel{\text{LH}}{=} \lim\limits_{x\to a} \frac{f^{(n)}(x)-T^{(n)}_{n,a}(x)}{n!} = 0\]

(c) \textit{Second Derivative Test Generalized}
Assume that \[f'(a)=f''(a)=\cdots=f^{(n-1)}(a)=0\] but $f^{(n)}(a)\neq0$.
\begin{enumerate}[label= (\roman*)]
    \item If $n$ is even and $f^{(n)}(a)$ is positive, then $f$ has a local minimum at $x=a$.
    \item If $n$ is even and $f^{(n)}(a)$ is negative, then $f$ has a local maximum at $x=a$.
    \item If $n$ is odd, then $f$ has neither a minimum or a maximum at $x=a$.
\end{enumerate}

Assume that $f(a)=0$. 

Notice that $f(x)=f(x)-f(a)$. This equation does not alter $f$ or its derivatives.

From the definition, \[T_{n,a}(x)=\sum_{k=0}^{n} \frac{f^{(k)}(a){(x-a)}^k}{k!}.\]

So,
\begin{align*}
    \frac{T_{n,a}(x)}{{(x-a)}^n}&=\frac{1}{{(x-a)}^n} \sum_{k=0}^{n} \frac{f^{(k)}(a){(x-a)}^k}{k!}\\
    &=\sum_{k=0}^{n} \frac{f^{(k)}(a){(x-a)}^{k-n}}{k!}.
\end{align*}

Recall that $f^{(k)}(a)=0$ for all $k<n$. So, our series reduces to,
\[\frac{f^{(n)}(a)}{n!}{(x-a)}^{n-n} = \frac{f^{(n)}(a)}{n!} = \frac{T_{n,a}(x)}{{(x-a)}^n}.\]

Then, we see that,
\[\frac{f(x)-T_{n,a}(x)}{{(x-a)}^n}=\frac{f(x)}{{(x-a)}^n}-\frac{f^{(n)}(a)}{n!}.\]

But, we have already seen that the left term goes to zero as $x$ approaches $a$. This informs us that the difference between the right terms must also go to zero. In other words, the term must be very close together.

We showed earier that \[\lim\limits_{x\to a} \frac{f(x)-T_{n,a}(x)}{{(x-a)}^n}=0\]

So, for all $\epsilon>0$, there is a $\delta$ such that for all $|x-a|<\delta$, then $\Big|\frac{f(x)-T_{n,a}(x)}{{(x-a)}^n}\Big|<\epsilon$.

Then, we also know that \[\Bigg|\frac{f(x)}{{(x-a)}^n}-\frac{f^{(n)}(a)}{n!}\Bigg|<\epsilon,\]
for all $|x-a|<\delta$.

Since $\frac{f(x)}{{(x-a)}^n}$ is $\epsilon$ close to the $n^{\text{th}}$ derivative of $f$ at $a$ when $x$ is $\delta$ close to $a$, we can conclude that this value shares the same sign as the $n^{\text{th}}$ derivative of $f$ at $a$.

We then proceed with (i) and (ii), where $n$ is even. 

For even $n$, ${(x-a)}^n$ will be positive for all $x\neq a$.
 
Also, for all positive $n$, $n!$ will be positive.

So, the ratio of these two terms, $\frac{{(x-a)}^n}{n!}$, will also be positive when $x \neq a$.

We will choose an $\epsilon$ such that $\epsilon<\Big|f^{(n)}\frac{{(x-a)}^n}{n!}\Big|$. For convenience, we will write $\epsilon_0 = \epsilon {(x-a)}^n$.

We see that,
\begin{align*}
    \frac{f^{(n)}(a)}{n!} - \epsilon < &\frac{f(x)}{{(x-a)}^n} < \frac{f^{(n)}(a)}{n!} + \epsilon \\
    f^{(n)}(a)\frac{{(x-a)}^n}{n!} - \epsilon_0 < &f(x) < f^{(n)}(a)\frac{{(x-a)}^n}{n!} + \epsilon_0,
\end{align*}
for all $|x-a|<\delta$

By our definition of epsilon, we see that in case (i), when $f^{(n)}(a)>0$ and $x \neq a$,
\[0<f^{(n)}(a)\frac{{(x-a)}^n}{n!} - \epsilon_0 < f(x),\]
so, \[f(x)>0.\]

Since $f(a)=0$ by our assumption, then with $f(x)>0$ for all $x\neq a$, we see that $f$ attains a local minimum at $x=a$.

In case (ii), with $f^{(n)}(a)<0$ and $x \neq a$, 
\[f(x)<f^{(n)}(a)\frac{{(x-a)}^n}{n!} + \epsilon_0 < 0,\] 
so \[f(x)<0.\]

Since $f(x)<0$ for all $x \neq a$, and $f(a)=0$, we see that $x=a$ is a local maximum for $f$.

For case (iii), when $n$ is odd, the sign of the ${(x-a)}^n$ term depends on $x$. For $x<a$, the term is negative; for $x>a$, the term is positive.

We return to our description of the proximity of $\frac{f(x)}{{(x-a)}^n}$ and $\frac{f^{(n)}(a)}{n!}$. We will first consider $x>a$ such that ${(x-a)}^n$ is positive. The steps follow from above, 
\begin{align*}
    \frac{f^{(n)}(a)}{n!} - \epsilon < &\frac{f(x)}{{(x-a)}^n} < \frac{f^{(n)}(a)}{n!} + \epsilon \\
    f^{(n)}(a)\frac{{(x-a)}^n}{n!} - \epsilon_0 < &f(x) < f^{(n)}(a)\frac{{(x-a)}^n}{n!} + \epsilon_0,
\end{align*} 
and we again see that the sign of values of $f(x)$ follows the sign of the $n^{\text{th}}$ derivative of $f$ at $a$. 

However, when $x<a$ and ${(x-a)}^n$ is negative, the identity flips,
\begin{align*}
    \frac{f^{(n)}(a)}{n!} - \epsilon < &\frac{f(x)}{{(x-a)}^n} < \frac{f^{(n)}(a)}{n!} + \epsilon \\
    f^{(n)}(a)\frac{{(x-a)}^n}{n!} - \epsilon_0 > &f(x) > f^{(n)}(a)\frac{{(x-a)}^n}{n!} + \epsilon_0.
\end{align*}
Similarly, we see that the sign of $f$ depends on $f^{(n)}(a)$, but now it does so inversely.

For $f^{(n)}(a)>0$ and $x \neq a$, again recalling that ${(x-a)}^n<0$,
\[0>f^{(n)}(a)\frac{{(x-a)}^n}{n!}>f(x)\]
implies that $f(x)<0$.

The same can be repeated for $f^{(n)}(a)$ negative.

Since the sign of $f^{(n)}(a)$ is fixed, and the sign of $f$ depends inversely on this term when $x<a$, but then directly on this term after $x>a$, we see that the sign of $f$ changes about $x=a$, which makes it neither a maximum nor a minimum.

\end{document}