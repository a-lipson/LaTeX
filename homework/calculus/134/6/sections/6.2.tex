\documentclass[../hw6.tex]{subfiles}

\begin{document}

\subsection*{10}
Sketch the region $\Omega$ bounded by the curves $y=x^2$ and $y=2-x$ and
find the volume of the solid generated by revolving this region
about the $x$-axis. 

Also, rotate about $y=-1$, $y=4$, and $y=\frac{1}{4}$.
% the last one is tricky; do not evaluate the integrals

In order sketch the region $\Omega$, we first need to find the intersection points.
\begin{align*}
    x^2&=2-x\\
    x^2+x-2&=0\\
    (x+2)(x-1)&=0\\
    x&=-2,1.
\end{align*}

\begin{figure*}[ht]
\centering
\begin{tikzpicture}
    \draw[->] (-3,0) -- (3,0) node[right] {$x$};
    \draw[->] (0,-1) -- (0,7) node[above] {$y$};
  
    \draw[domain=-2.5:2.5, smooth, variable=\x, blue] plot ({\x}, {\x*\x}) node[right] {$y=x^2$};
    \draw[domain=-2.5:2.5, smooth, variable=\x, red] plot ({\x}, {2-\x}) node[right] {$y=2-x$};

    \draw (-2,0.1) -- (-2,-0.1) node[below] {-2};
    \draw (1,0.1) -- (1,-0.1) node[below] {1};

    \draw[dashed] (-2,0) -- (-2,4);
    \draw[dashed] (1,0) -- (1,1);

    \node at (-0.5,1.5) {$\Omega$};
\end{tikzpicture}
\end{figure*}

Then, to find the volume of the solid of revolution by the $x$-axis, we note that the outer function is $y=2-x$ and the inner is $y=x^2$.
We integrate with the bounds given by the intersection of the curves above.
\begin{align*}
    V_0 &= \pi \int_{-2}^{1} {(2-x)}^2 - {(x^2)}^2 dx \\
    &= \pi \int_{-2}^{1} 4-4x+x^2-x^4 dx \\
    &= \pi \left[ 4x-2x^2+\frac{x^3}{3}-\frac{x^5}{5} \right] \Bigg\vert_{-2}^{1} \\
    &= \frac{\pi}{15} \left[ 60x-30x^2+5x^3-3x^5 \right] \Big\vert_{-2}^{1} \\
    &= \frac{\pi}{15} \left[ (60-30+5-3) - (-120-120-40+96) \right] \\
    &= \frac{\pi}{15} \left[ 32 - (-184) \right] \\
    &= \frac{216 \pi}{15} \\
    &= \frac{72 \pi}{5}. \\
\end{align*}

For rotating about $y=-1$, we take each of the functions shifted up by one unit, $y=2-x+1$ and $y=x^2+1$. The line is the outer function and the parabola is the inner function.
\begin{align*}
    V_{-1} &= \pi \int_{-2}^{1} {(2-x+1)}^2 - {(x^2+1)}^2 dx \\
    &= \pi \int_{-2}^{1} 9-6x+x^2-x^4-2x^2-1 dx \\
    &= \pi \int_{-2}^{1} -x^4-x^2-6x+8 dx \\
    &= \pi \left[ \frac{-x^5}{5}-\frac{x^3}{3}-3x^2+5x \right] \Bigg\vert_{-2}^{1} \\
    &= \pi \left( \left( \frac{-1}{5}-\frac{1}{3}-3+8 \right) - \left( \frac{32}{5}+\frac{8}{3}-12-16 \right) \right) \\
    &= \pi \left( \frac{-33}{5}-\frac{9}{3}+33 \right) \\
    &= \pi \left( \frac{150}{5} - \frac{33}{5} \right) \\
    &= \frac{117 \pi}{5}. \\
\end{align*}

When rotating about $y=4$, we note that the functions are shifted downward by four, and that the parabola is the outer function while the line is the inner function.
\begin{align*}
    V_4 &= \pi \int_{-2}^{1} {(x^2-4)}^2-{(-x-2)}^2 dx \\
    &= \pi \int_{-2}^{1} \left( x^4-8x^2+16 \right) - \left( x^2+4x+4 \right) dx \\
    &= \pi \int_{-2}^{1} x^4-9x^2-4x+12 dx \\
    &= \pi \left[ \frac{x^5}{5}-3x^3-2x^2+12x \right] \Bigg\vert_{-2}^{1} \\
    &= \pi \left( \left( \frac{1}{5}-3-2+12\right) - \left( \frac{-32}{5}+24-8-24 \right) \right) \\
    &= \pi \left( \frac{33}{5}+15 \right) \\
    &= \frac{108 \pi}{5}. \\
\end{align*}

Now, when rotating about $y=\frac{1}{4}$, we notice that the axis of rotation intersects the region. Thus, the part of the region underneath the axis will have a negative signed area. If we integrate the region as a solid of rotation, we will obtain a cavity in the final volume. To remedy this, we simply add back a positive portion of the solid of rotation obtained by the part of the region that is below the axis.

The axis $y=\frac{1}{4}$ will only intersect region when it is defined by the parabola. 
The roots of the parabola are found by $x^2-\frac{1}{4}=0$, which are $x=\pm \frac{1}{2}$. 

The root of the line with the axis is $2-x-\frac{1}{4}=0$, which is $x=\frac{9}{4}$. This $x$ value lies outside the $x$ bounds of the region $\Omega$, so we know that $x=-\frac{1}{2}$, and $x=\frac{1}{2}$ are the bounds of the cavity.

So, $y=2-x-\frac{1}{4}=\frac{7}{4}-x$ is the outer function and $y=x^2-\frac{1}{4}$ is the inner function.

Thus, our final volume integral is given by the solid of revolution with a cavity combined with the volume to fill the cavity, 
\[V_{\frac{1}{4}} = \pi \int_{-2}^{1} {\left( \frac{7}{4} - x\right)}^2 - {\left( x^2-\frac{1}{4} \right)}^2 dx + \pi \int_{-\frac{1}{2}}^{\frac{1}{2}} {\left( x^2-\frac{1}{4} \right)}^2 dx.\]


\subsection*{44}
A hemispherical punch bowl 2 feet in diameter is filled to
within 1 inch of the top. Thirty minutes after the party starts,
there are only 2 inches of punch left at the bottom of the bowl.

(a) How much punch was there at the beginning?

(b) How much punch was consumed?

To find the volume of a sphere of radius 12, we will consider the solid of rotation of a circle.

We center the circle at the origin, and thus our punch bowl is defined by the curve, $y=-\sqrt{12^2-y^2}=-\sqrt{144-y^2}$, where $y$ is below the $x$-axis to represent the bowl.

For (a), we integrate up to one inch below the $x$-axis, which is the top of the bowl.

With the interval $y=-12$, the bottom of the bowl, given by the radius, to $y=-1$, we integrate a solid of rotation to determine the initial volume of punch.
\begin{align*}
    V_i &= \pi \int_{-12}^{-1} {\sqrt{144-y^2}}^2 dy \\
    &= \pi \int_{-12}^{-1} 144-y^2 dy \\
    &= \pi \left[ 144y-\frac{y^3}{3} \right] \Bigg\vert_{-12}^{-1}\\
    &= \pi \left( \left( -144+\frac{1}{3} \right) - \left( -1728-\frac{-1728}{3} \right)\right) \\
    &= \pi \left( -\frac{431}{3} - \left( -\frac{3456}{3} \right) \right) \\
    &= \frac{3025\pi}{3}. \\
\end{align*}

Similarly, for the final volume of punch, the interval is $y\in[-12,-10]$, where only two inches of punch remain in the bowl, so $-12+2=-10$.
\begin{align*}
    V_f &= \pi \int_{-12}^{-10} {\sqrt{144-y^2}}^2 dy \\
    &= \pi \int_{-12}^{-10} 144-y^2 dy \\
    &= \pi \left[ 144y-\frac{y^3}{3} \bigg\vert_{-12}^{-10} \right] \\
    &= \pi \left( \left( -1440+\frac{1000}{3} \right) - \left( -1728-\frac{-1728}{3} \right)\right) \\
    &= \pi \left( \left( -\frac{3320}{3} \right) - \left( -\frac{3456}{3} \right) \right) \\
    &= \pi \left( \frac{136}{3} \right) \\
    &= \frac{136\pi}{3}. \\
\end{align*}

So, for (b), the volume of punch consumed, being the difference between the initial volume and the final volume, is,
\[\frac{3025\pi}{3} - \frac{136\pi}{3} = \frac{2889\pi}{3} = 963\pi,\]
or about 3025 cubic inches of punch.

This volume, is roughly equivalent to 13.1 gallons or about 1676 fluid ounces. 

Considering that one drink is about 12 ounces, then a total of $\frac{1676}{12}\approx140$ drinks were consumed in 30 minutes.

Assuming that the number of persons who didn't drink at all is equivalent to the number of persons who had two drinks within 30 minutes, we guess that the attendance of this party was roughly 140 persons.

\end{document}