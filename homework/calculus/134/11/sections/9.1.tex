\documentclass[../hw11]{subfiles}

\begin{document}

\subsection*{30}
Find the general solution of $y'+ry=0$, $r$ constant. 

\begin{enumerate}[label= (\alph*)]
    \item Show that if $y$ is a solution and $y(a)=0$ at some number $a\geq0$, then $y(x)=0$ for all $x$. (Thus a solution y is either identically zero or never zero.)
    \item Show that if $r<0$, then all nonzero solutions are unbounded.
    \item Show that if $r>0$, then all solutions tend to $0$ as $x\to\infty$.
    \item What are the solutions if $r=0$?
\end{enumerate}

First, we compute the general solution to the first order linear homogeneous differential equation by separation of variables.
\begin{align*}
    \frac{dy}{dx}+ry&=0\\
    \int \frac{dy}{y} &= \int -r\,dx \\
    \ln{|y|}&=c_0-rx\\
    |y|&=e^{c_0-rx} \\
    |y|&=e^{c_0}e^{-rx} \\
    y&=\pm e^{c_0}e^{-rx} \\
    y&=c_1 e^{-rx}.
\end{align*}

The general solution is \[y=ce^{-rx}.\]

For (a), we will show that $y$ is either the trivial solution $y(x)=0$, or $y$ is never zero.

If, for some $a\geq0$, $y(a)=0$, then, $0=ce^{-rx}$.

But, $e^b>0$ for all $-rx = b \in \mathbb{R}$.

Then $ce^{-rx}=0$ implies that $c=0$.

So $y(x)=0\cdot e^{-rx}=0$ for all $x$.

If $c\neq 0$ and $e^{-rx}\neq0$ for all $x$, then their product $f(x)\neq0$ either.

For (b), if $r<0$, then $-r>0$.

If $c=0$, then, by (a), the solution is identically zero, so it does grow without bound. So, we continue with $c\neq0$.

So, when $x$ exceeds any number, with $0<b=-r$ $\lim\limits_{x\to\infty} ce^{bx}$ grows without bound.

Thus, when $r<0$, the nontrivial solutions $y$ are unbounded.

For (c), we note that the trivial solution is zero everywhere, so it tends toward zero as $x$ exceeds any number.

As $x$ exceeds any number, with $r>0$, $\lim\limits_{x\to\infty} ce^{rx}$ will become unbounded by (b).

We note that $y=ce^{-rx}=\frac{1}{ce^{rx}}$.

So, $\lim\limits_{x\to\infty} \frac{1}{ce^{rx}}$ will have is denominator grow without bound and thus will tend toward zero as $x$ exceeds any number.

Therefore for all solutions with $r>0$, $y$ tends toward zero.

For (d), when $r=0$, we see that $y=ce^{-rx}=c$.

So, the solution is uniquely $y=c$ when $r=0$.


\subsection*{31}
Consider the differential equation \[y'+p(x)y=0\] with $p$ continuous on an interval $I$.
\begin{enumerate}[label= (\alph*)]
    \item Show that if $y_1$ and $y_2$ are solutions, then $u=y_1+y_2$ is also a solution.
    \item Show that if $y$ is a solution and $C$ is a constant, then $u=Cy$ is also a solution.
\end{enumerate}

For (a), since $y_1$ and $y_2$ are both solutions, we are given that
\begin{align*}
    {y_1}'+p(x)y_1&=0,\\
    {y_2}'+p(x)y_2&=0.\\
\end{align*}

We take their sum,
\[({y_1}'+{y_2}')+p(x)(y_1+y_2)=0.\]

Given that the sum of a derivative is the same is the derivative of a sum, we get
\[{(y_1+y_2)}'+p(x)(y_1+y_2)=0.\]

When we let $u=y_1+y_2$, we see that $u$ is also a solution to the differential equation,
\[u'+p(x)u=0.\]

For (b), we begin with the solution $y$ where
\[y'+p(x)y=0.\]

We multiply by the arbitrary constant $C$, 
\begin{align*}
    C(y'+p(x)y)&=C\cdot0\\
    Cy'+p(x)Cy&=0.\\
\end{align*}

We note that the derivative of a constant times a function is the constant times the derivative of the function. So, we get
\[(Cy)'+p(x)Cy=0.\]

We define the function $u=Cy$ where $y$ is the given solution and see that,
\[u'+p(x)u=0.\]

So, $u$ is also a solution to the differential equation.

% First, we find the general solution to the differential equation by separation of variables. We will define $P(x)$ as the antiderivative of $p(x)$.
% \begin{align*}
%     \frac{dy}{dx}+p(x)y&=0\\
%     \int \frac{dy}{y}&=\int -p(x)\,dx\\
%     \ln{|y|}&=c_0-P(x)\\
%     |y|&=e^{c_0} e^{-P(x)}\\
%     y&=\pm e^{c_0} e^{-P(x)}\\
%     y&=ce^{-P(x)}.\\
% \end{align*}

% For (a), we will demonstrate the superposition principle by considering the two unique solutions
% \begin{align*}
%     y_1 &= c_1 e^{-P(x)},\\
%     y_2 &= c_2 e^{-P(x)}.\\
% \end{align*}

% Then, their sum $u=y_1+y_2$ is
% \begin{align*}
%     y_1+y_2&=c_1 e^{-P(x)}+c_2 e^{-P(x)} \\
%     &= (c_1+c_2)e^{-P(x)} \\
%     &= c_3 e^{-P(x)}, \\
% \end{align*}
% which is another unique solution to the differential equation.

% For (b), we will demonstrate that multiplication by a constant does not impair the validity of a solution.

% Let $u=Cy$ where $y=c_1 e^{-P(x)}$ is a solution to the differential equation.

% So,
% \begin{align*}
%     u&=Cy \\
%     &= C\cdot c_1 e^{-P(x)} \\
%     &= c_4 e^{-P(x)}, \\
% \end{align*}
% which is another unique solution to the differential equation. 


\subsection*{34}

Show that if $y_1$ and $y_2$ are solutions to \[y'+p(x)y=q(x),\] then $y=y_1-y_2$ is a solution to $y'+p(x)y=0$.

Since $y_1$ and $y_2$ are both solutions, we see that
\begin{align*}
    {y_1}'+p(x)y_1&=q(x),\\
    {y_2}'+p(x)y_2&=q(x).\\
\end{align*}

We take their difference,
\[{y_1}'-{y_2}'+p(x)(y_1-y_2)=0.\]

We note, by linearity of the derivative, the above becomes
\[{(y_1-y_2)}'+p(x)(y_1-y_2)=0.\]

We define $y=y_1-y_2$ and see that,
\[y'+p(x)y=0.\]

So, $y$ is a solution to the equation $y'+p(x)y=0$.

% First, by the previous problem, we note that $y=ce^{-P(x)}$ where $P(x)$ is the antiderivative of $p(x)$ is a solution to the latter differential equation $y'+p(x)y=0$.

% We solve the given first order linear nonhomogeneous differential equation by using the integrating factor $e^{\int p(x)\,dx}=e^{P(x)}$.

% So,
% \begin{align*}
%     e^{P(x)}y'+e^{P(x)}p(x)y &= e^{P(x)}q(x) \\
%     \frac{d}{dx}\left[ e^{P(x)}y \right]&=e^{P(x)}q(x) \\
%     e^{P(x)}y&=\int e^{P(x)}q(x)\,dx \\
%     y&=e^{-P(x)}\int e^{P(x)}q(x)\,dx. \\
% \end{align*}

% We verify this by comparing to the given solution from 9.1.2,
% \[y(x)=e^{-P(x)}\left[ \int e^{P(x)}q(x)\,dx + C \right].\]

% We note that the extra constant is a result of evaluating the integral.

% We define two unique solutions
% \begin{align*}
%     y_1&=e^{-P(x)}\left[ \int e^{P(x)}q(x)\,dx+C_1 \right], \\
%     y_2&=e^{-P(x)}\left[ \int e^{P(x)}q(x)\,dx+C_2 \right]. \\
% \end{align*}

% We compute their difference,
% \begin{align*}
%     y_1-y_2&=e^{-P(x)}\left[ \left[ \int e^{P(x)}q(x)\,dx+C_1 \right] - \left[ \int e^{P(x)}q(x)\,dx+C_2 \right] \right] \\
%     &=e^{-P(x)}(C_1-C_2) \\
%     &=C_3 e^{-P(x)}, \\
% \end{align*}

% Which is also a valid solution to the homogeneous differential equation $y'+p(x)y=0$.


\end{document}