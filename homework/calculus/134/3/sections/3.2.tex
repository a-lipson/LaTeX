\documentclass[../hw3.tex]{subfiles}

\begin{document}

\subsection*{50}
Find $A$ and $B$ given that the derivate of $f(x) = \begin{cases}
    Ax^2+B, & x<-1 \\
    Bx^5+Ax+4, & x\geq-1
\end{cases}$ is everywhere continuous.

\begin{proof}
    Since $f$ is differentiable, it is continuous.

    Therefore, given that $f$ is continuous,
    \begin{align*}
        \lim\limits_{x \to -1^-} Ax^2+B &= \lim\limits_{x \to -1^+} Bx^5+Ax+4 \\
        A+B &= -B-A+4 \\
        2B &= 4-2A \\
        B = 2-A \\
    \end{align*}

    Then, we find $f'(x)$.
    \[f(x) = \begin{cases}
        2Ax & x<-1 \\
        5Bx^4 + A & x\geq-1 \\
    \end{cases}\]

    Since $f'$ is continuous, then, \[\lim\limits_{x \to -1^-} f' = \lim\limits_{x \to -1^+} f' \text{and} \lim\limits_{x \to c} f'(x) = f'(c). \]

    So,
    \begin{align*}
        \lim\limits_{x \to -1^-} 2Ax &= \lim\limits_{x \to -1^+} 5Bx^4+A \\
        -2A &= 5B+A \\
        -3A &= 5B \\
        \frac{-3}{5}A &= B \\
    \end{align*}

    Then, using the relationship $B=2-A$ derived earlier, we can solve the system of equations.
    \begin{align*}
        2-A &= \frac{-3}{5}A \\
        2 &= \frac{2}{5}A \\
        5 &= A \\
    \end{align*}

    Lastly, \[B=2-A=2-5=-3.\]

\end{proof}

\subsection*{62}
Set $f(x)=x^3$.

(a) Find an equation for the line tangent to the graph of $f$ at $(c,f(c)),\quad c \neq 0$.

(b) Determine whether the tangent line found in (a) intersects the graph of $f$ at a point other than $(c,c^3)$.

\begin{proof}[Proof of (a).]
    \textbf{Definition 3.1.2.} Let the tangent line to a point $(c,f(c))$ be given by the equation, \[y-f(c)=f'(c)(x-c).\]

    Let $f(c) = c^3$.
    
    Then, $f'(c) = 3c^2$ by 3.2.7.

    Next, we use $f'$ to find the tangent line equation at $(c,f(c))$,
    \begin{align*}
        y-c^3 &= 3c^2 (x-c) \\
        y &= 3xc^2 - 3c^3 + c^3 \\
        y &= 3xc^2 - 2c^3 \\
    \end{align*}

    Therefore, the tangent line at $(c,f(c))$ is $y = 3xc^2-2c^3$.
\end{proof}

\begin{proof}[Proof of (b).]
    We must check if the tangent line $y$ found in (a) intersects the given curve $f(x) = x^3$.

    So, we will find all the $(x,x^3)$ on $f$ that satisfy $y = 3xc^2-2c^3$
    \begin{align*}
        x^3 &= 3xc^2-2c^3 \\
        x^3-3xc^2+2c^3 &= 0 \\
    \end{align*}

    We can begin to factor this polynomial using the fact that $y$ is a tangent line to $f$ at $x=c$ given the intersection point $(c,c^3)$. Since a tangent line only touches the curve $f$ but does not pass through it, it can be considered a second degree root of this equation. For this reason, we will begin to factor by ${(x-c)}^2$.
   
    We will show that the resulting zero is $(2c+x)$.
    \begin{align*}
        x^3-3xc^2+2c^3 &= {(x-c)}^2 (2c+x) \\
        &= (x^2-2xc+c^2)(2c+x) \\
        &= (2cx^2-4xc^2+2c^3+x^3-2cx^2+xc^2) \\
        &= x^3-3xc^2+2c^3
    \end{align*}

    With $(2c+x) = 0$, we see that $x={-2}c$.

    So, the valid $x$ are $x = c$ and $x = {-2}c$.

    With the first $x$, we reobtain the given point $(c,c^3a)$.

    However, with the latter $x$, we get the second intersection point on $f$,
    \[({-2}c,{-8}c^3).\] 
\end{proof}


\end{document}