\documentclass[../hw3.tex]{subfiles}

\begin{document}

\subsection*{62}
Let $f$ be a differentiable function. Use the chain rule to show that:

(a) if $f$ is even, then $f'$ is odd.

(b) if $f$ is odd, then $f'$ is even.

Assume that an even function $g$ defined such that it satisfies the property $g(x) = g(-x)$.

Assume that an odd function $g$ defined such that it satisfies the property $g(x) = -g(-x)$.

\begin{proof}[Proof of (a).]
    We differentiate both sides with respect to $x$.
    \begin{align*}
        \frac{d}{dx} \left[ f(x) \right] &= \frac{d}{dx} \left[ f(-x) \right] \\
        f'(x) &= -f'(-x) \\
    \end{align*}

    We recognize this to be the form of an odd function where $f' = g$.
\end{proof}

\begin{proof}[Proof of (b).]
    We differentiate both sides with respect to $x$.
    \begin{align*}
        \frac{d}{dx} \left[ f(x) \right] &= \frac{d}{dx} \left[ -f(-x) \right] \\
        f'(x) &= f'(-x) \\
    \end{align*}

    We recognize this to be the form of an even function where $f' = g$.
\end{proof}

\end{document}