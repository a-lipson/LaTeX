\documentclass[../m134a-hw2.tex]{subfiles}

\begin{document}
\subsection*{38}
For each of the limits stated and the $\epsilon$'s given, use a graphing utility to find a $\delta>0$ which is such that if $0<|x-c|<\delta$, then $|f(x)-L|<\epsilon$.
Draw the graph of $f$ together with the vertical lines $x=c-\delta, x=c+\delta$ and the horizontal lines $y=L-\epsilon, y=L+\epsilon$

$\lim\limits_{x \to 0} (2 - 5x) = 2$.

\begin{figure*}[ht]
\centering
\begin{tikzpicture} % lim x->0 2-5x = 2
    \draw[->] (0,-0.5) -- (0,5.5);
    \draw[->] (-0.5,0) -- (5.5,0);

    \draw[<->, blue, thick] (-0.5,4.5) -- (4.5,-0.5);

    \node[left] at (0,2) {$L$};
    \node[below] at (2,0) {$c$};
    \draw[thick] (-0.1,2) -- (0.1,2);
    \draw[thick] (2,-0.1) -- (2,0.1); 

    \draw[<->] (1.5,0.25) -- (2,0.25);
    \draw[<->] (2.5,0.25) -- (2,0.25);
    \node[above] at (1.75,0.25) {$\delta$};
    \node[above] at (2.25,0.25) {$\delta$};
    \draw[dashed] (1.5,0) -- (1.5,2.5);
    \draw[dashed] (2.5,0) -- (2.5,1.5);

    \draw[<->] (0.25,1.5) -- (0.25,2);
    \draw[<->] (0.25,2.5) -- (0.25,2);
    \node[right] at (0.25,1.75) {$\epsilon$};
    \node[right] at (0.25,2.25) {$\epsilon$};
    \draw[dashed] (0,2.5) -- (1.5,2.5);
    \draw[dashed] (0,1.5) -- (2.5,1.5);

\end{tikzpicture}
\end{figure*}

\begin{proposition}
    For every $\epsilon > 0$, there exists a $\delta > 0$ such that $|x-0|<\delta$ implies $|2-5x-2|<\epsilon$.
\end{proposition}

\begin{proof}
    Let $\epsilon > 0$ be given.
    
    Define $\delta = \frac{\epsilon}{5}$
    
    Assume $|x| < \delta$, then
    \begin{align*}
        |x| &< \frac{\epsilon}{5} \\
        5|x| &< \epsilon \\
        |5x| &< \epsilon \\
        |{-5}x| &< \epsilon \\
        |{-5}x+2-2| &< \epsilon \\
        |2-5x-2| &< \epsilon \\
    \end{align*}

    Hence, $\lim\limits_{x \to 0} (2 - 5x) = 2$.

\end{proof}


\subsection*{52}
Give an $\epsilon-\delta$ proof for $\lim\limits_{x \to 3} \sqrt{x+1} = 2$.

\textbf{Theorem 2.3.2.}
\textit{$\lim\limits_{x \to c} f(x) = M$ and $\lim\limits_{x \to M} g(x) = L$ implies $\lim\limits_{x \to c} g(f(x)) = L$.}
% \begin{theorem}
% \end{theorem}

\begin{proposition}
    For every $\epsilon > 0$, there exists a $\delta > 0$ such that $|x-3|<\delta$ implies $|\sqrt{x+1}-2|<\epsilon$.
\end{proposition}

\begin{proof}
    We will split this limit into a composition of two limits, where $f(x) = x + 1$ and $g(x) = \sqrt{x}$ such that $g(f(x)) = \sqrt{x+1}$ by Theorem 2.3.2. 

    First, since $x+1$ is polynomial, then $x+1$ is continuous. Therefore, $\lim\limits_{x \to c} f(x) = f(c)$.
    
    So, $\lim\limits_{x \to 3} x+1 = 4$.

    For an $\epsilon-\delta$ proof, we make the assumption that, for every $\epsilon > 0$, there exists a $\delta > 0$ such that $|x-3|<\delta$ implies $|x+1-4|<\epsilon$.

    Let $\epsilon>0$ be given.
    Define $\delta=\epsilon$.
    Assume $|x-3|<\delta$.
    
    Then, $|x-3|=|x+1-4|<\epsilon$.

    So, $\lim\limits_{x \to 3} x+1 = 4$.

    Second, for $\lim\limits_{x \to 4} \sqrt{x} = 2$, for every $\epsilon>0$, there exists a $\delta>0$ such that $|x-4|<\delta$ implies $|\sqrt{x}-2|<\epsilon$.

    Let $\epsilon>0$ be given.

    Choose $\delta = \min\{4,\epsilon\}$.

    Assume $0<|x-4|<\delta$.

    Since $\delta\leq4$, then $x\geq0$.
    So, $\sqrt{x}$ is defined.
    \begin{align*}
        x-4 &= {\sqrt{x}}^2-2^2 \\
        x-4 &= (\sqrt{x}+2)(\sqrt{x}-2) \\
        |x-4| &= \left|(\sqrt{x}+2)(\sqrt{x}-2)\right| \\
        |x-4| &= |\sqrt{x}+2||\sqrt{x}-2| \\
    \end{align*}
    Since $\sqrt{x}+2\geq2>1$, then $|x-4|>|\sqrt{x}-2|$.

    Since $|x-4|<\delta$ and $\delta\leq\epsilon$, then $|\sqrt{x}-2|<\epsilon$.

    Therefore, $\lim\limits_{x \to 4} = 2$.

    Since $\lim\limits_{x \to 3} x+1 = 4$ and $\lim\limits_{x \to 4} = 2$, then $\lim\limits_{x \to 3} \sqrt{x+1} = 2$ by Theorem 2.3.2.

\end{proof}

\subsection*{54}

Prove that, for the function \[g(x) = \begin{cases} 
    x, & \text{$x$ rational} \\
    0, & \text{$x$ irrational}, \\
\end{cases}\]
$\lim\limits_{x \to 0} g(x) = 0$.

\begin{proposition}
    For all $\epsilon>0$, there exists a $\delta>0$ such that $|x-0|=|x|<\delta$ implies $|g(x)-0|=|g(x)|<\epsilon$.
\end{proposition}

\begin{proof}
    Let $\epsilon>0$ be given.
    Define $\delta=\epsilon$.
    Assume $0<|x|<\delta$.
    
    Then, there are two cases, for $x \in \mathbb{R}$, $x$ is either rational or irrational.

    In the first case, where $x \in \mathbb{Q}$, $|g(x)|=|x|<\delta=\epsilon$.

    So, $|g(x)|<\epsilon$ and the statement holds for the first case.

    In the second case, when $x \in \mathbb{R} \setminus \mathbb{Q}$, $|g(x)|=|0|=0$. 
    
    Since $0<\epsilon$ was given, $|g(x)|<\epsilon$ also holds in the second case.

    Therefore, $\lim\limits_{x \to 0} g(x) = 0$.
\end{proof}

\end{document}

