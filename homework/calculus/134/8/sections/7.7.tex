\documentclass[../hw8]{subfiles}

\begin{document}
A person walking along a straight path at the rate of 6 feet
per second is followed by a spotlight that is located 30 feet
from the path. How fast is the spotlight turning at the instant
the person is 50 feet past the point on the path that is closest
to the spotlight?

\begin{figure*}[ht]
\centering
\begin{tikzpicture}

    \draw[gray, ->] (-1,0) -- (6,0) node[right] {$x$};
    % Define the triangle
    \coordinate (A) at (0,0);
    \coordinate (B) at (5,0);
    \coordinate (C) at (0,3);

    % Draw the triangle
    \draw (A) -- (B) -- (C) -- cycle;

    \node at (4,2) {$\frac{dx}{dt}=6$};

    % Label the vertices
    \fill (C) circle[radius=2pt];

    % Label the sides
    \node[below] at ($(A)!0.5!(B)$) {$x=50$};
    \node[left] at ($(A)!0.5!(C)$) {$30$};

    % Mark the angle theta
    \pic [draw, angle radius=0.4cm, "$\theta$", angle eccentricity=1.5] {angle = A--C--B};
\end{tikzpicture}
\end{figure*}

From the figure, we can see that
\begin{align*}
    \tan{\theta} &= \frac{x}{30} \\
    \theta &= \arctan{\frac{x}{30}},
\end{align*}

So, with the given conditions of $x=50$ and $\frac{dx}{dt}=6$,
\begin{align*}
    \frac{d\theta}{dt} &= \frac{1}{{\left( \frac{x}{30} \right)}^2+1} \left( \frac{1}{30} \right) \left( \frac{dx}{dt} \right) \\
    &= \frac{1}{{\left( \frac{5}{3} \right)}^2+1} \left( \frac{1}{5} \right) \\
    &= \frac{1}{5\left( \frac{25}{9} + 1 \right)} \\
    &= \frac{1}{5\left( \frac{34}{9} \right)} \\
    &= \frac{9}{170}. \\
\end{align*}

So, at 50 feet past the perpendicular of the spotlight on the path, $\frac{d\theta}{dt}=\frac{9}{170}$ radians per second.


\end{document}