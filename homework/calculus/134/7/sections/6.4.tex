\documentclass[../hw7.tex]{subfiles}

\begin{document}

\subsection*{30}
Find the volume of the solid generated by revolving the triangular region in the figure. 

\begin{figure*}[ht]
\centering
\begin{tikzpicture}[scale=0.5]
    \draw[->] (0,-1) -- (0,4) node[above] {$y$};
    \draw[->] (-1,0) -- (5,0) node[right] {$x$};

    \draw[thick] (4,0.1) -- (4,-0.1) node[below] {$b$};
    \draw[thick] (1.5,0.1) -- (1.5,-0.1) node[below] {$a$};
    \draw[thick] (0.1,3) -- (-0.1,3) node[left] {$h$};

    \draw (0,0) -- (1.5,3) -- (4,0);
    \draw (1.5,0) -- (1.5,3);
\end{tikzpicture}
\end{figure*}

(a) about the x-axis.

(b) about the y-axis.

For (a), we start by finding two function to represent the hypotenuses of the triangles.

Let the hypotenuse segment with positive slope be defined by the function $f$, while the segment with negative slope be defined by $g$.

The slope of $f$ with respect to the vertical axis is $\frac{a}{h}$. Since $f$ intersects the origin, then $f(y) = \frac{a}{h}y$.

The slope of $g$ with respect to $y$ is $\frac{a-b}{h}$. It intercepts the $x$-axis at $x=b$. So, $g(y)=\frac{a-b}{h}y+b$.

We integrate around the $x$-axis with cylindrical shells. 

The circumference of these shells is given by $2\pi y$, where $y$, the distance from the $x$-axis, is the shell's radius. 

The height of each shell is given by the difference between $f$ and $g$. Since, $g>f$ on $[0,h]$, then we will take $g$ to define the upper part of the cylindrical shell to retain a positive area.

So, the height is given by,
\begin{align*}
    g(y)-f(y) &= \frac{a-b}{h}y+b-\frac{a}{h}y \\
    &= b - \frac{b}{h}y \\
    &= b\left( 1-\frac{y}{h} \right).
\end{align*}

The thickness of each cylindrical shell is the differential $\Delta y$.

We then integrate the volume of shells over $[0,h]$,
\begin{align*}
    &\int_{0}^{h} 2\pi y \cdot b\left( 1-\frac{y}{h} \right)\,dy \\
    &= 2\pi b \int_{0}^{h} y-\frac{y^2}{h}\,dy \\
    &= 2\pi b \left[ \frac{y^2}{2} - \frac{y^3}{3h} \right] \bigg\vert_{0}^{h} \\
    &= 2\pi b \left( \frac{h^2}{2} - \frac{h^3}{3h} \right) \\
    &= 2\pi b \left( \frac{h^2}{6} \right) \\
    &= \frac{\pi bh^2}{3}. \\
\end{align*}

So, the solid of revolution about the $x$-axis has an area of $\frac{\pi bh^2}{3}$.

For (b), we will integrate about the $y$-axis with washers, where $g(y)$ is further from the axis of rotation. So, we take $g(y)$ to be the outer function.

We integrate the washers over the same interval, $[0,h]$,
\begin{align*}
    & \pi \int_{0}^{h} {[g(y)]}^2-{[f(y)]}^2dy \\
    &= \pi \int_{0}^{h} {\left( \frac{a-b}{h}y + b \right)}^2 - {\left( \frac{a}{h}y \right)}^2 \, dy \\
    &= \pi \int_{0}^{h} \frac{{(a-b)}^2}{h^2}y^2+\frac{2b(a-b)}{h}y+b^2-\frac{a^2}{h^2}y^2\,dy \\
    &= \pi \int_{0}^{h} \frac{a^2-2ab+b^2}{h^2}y^2+\frac{2b(a-b)}{h}y+b^2-\frac{a^2}{h^2}y^2\,dy \\
    &= \pi \int_{0}^{h} \frac{b^2-2ab}{h^2}y^2+\frac{2b(a-b)}{h}y+b^2\,dy \\
    &= \pi b \left[ \frac{b-2a}{3h^2}y^3+\frac{a-b}{h}y^2+by \right] \Bigg\vert_{0}^{h} \\
    &= \pi b \left( \frac{b-2a}{3h^2}h^3+\frac{a-b}{h}h^2+bh \right) \\
    &= \pi b \left( \frac{(b-2a)h}{3} + (a-b)h + bh \right) \\
    &= \pi bh \left( \frac{b-2a}{3}+a-b+b\right) \\
    &= \pi bh \left( \frac{b+a}{3} \right) \\
    &= \frac{\pi}{3}bh(a+b). \\
\end{align*}

So, the volume of the solid of revolution about the $y$-axis is $\frac{\pi}{3}bh(a+b)$.

\subsection*{Project}
If a solid is \textit{homogeneous} (constant mass density), then the center of mass depends only on the shape of the solid and is called the \textit{centroid}. 
In general, determination of the centroid of a solid requires triple integration.
However, if the solid is a solid of revolution, then the centroid can be found by one-variable integration.

\begin{figure*}[ht]
\centering
\begin{tikzpicture}
    \draw[->] (-1,0) -- (5,0) node[right] {$x$};
    \draw[->] (0,-1) -- (0,3) node[above] {$y$};

    \draw[blue, name path=B] (1,2) to[bend right=40] (4,2.5);

    \draw (1,2) -- (1,0) node[below] {$a$};
    \draw (4,2.5) -- (4,0) node[below] {$b$};
    \node[blue] at (2,2) {$f$};
    \node at (2.5,1) {$\Omega$};

    \path[name path=A] (1,0) -- (4,0);
    \tikzfillbetween[of=A and B, on layer=main]{blue, opacity=0.3};
    
\end{tikzpicture} 
\end{figure*}

Explain Problems 1 and 2 briefly with a picture.

\subsubsection*{1}
Let $\Omega$ be the region shown in the figure and let T be the solid generated by revolving $\Omega$ around the x-axis. 
By symmetry, the centroid of $T$ is on the $x$-axis. Thus the centroid of $T$ is determined solely by its $x$-coordinate $\overline{x}$.

Show that $\overline{x}V=\int_{a}^{b} \pi x{[f(x)]}^2 dx$ where $V$ is the volume of $T$.

Use the following principle: if a solid of volume V consists of a finite number of pieces with volumes $V_1,V_2,\ldots,V_n$ and the pieces have centroids $x_1,x_2,\ldots,x_n$, then $\overline{x}V = x_1 V_1+x_2 V_2+\cdots+x_n V_n$.

\begin{figure*}[ht]
\centering
\begin{tikzpicture}
    \draw[gray, ->] (-1,0) -- (5,0) node[right] {$x$};
    \draw[gray, ->] (0,-2) -- (0,3) node[above] {$y$};

    \draw[blue] (1,2) to[bend right=40] 
        coordinate[pos=0.4] (A)
        coordinate[pos=0.45] (B) (4,2.5);

    \path let \p_=(A) in 
        coordinate (Ax0) at (\x_+3,0)
        coordinate (Ay) at (0,\y_);

    \draw (1,2) -- (1,0) node[below] {$a$};
    \draw (4,2.5) -- (4,0) node[below] {$b$};

    \def\arcflip #1 #2 #3 {
        \path let \p_=(#1) in coordinate (X) at (\x_,0) coordinate (Y) at (0,\y_); % can't use let binding with coordinate
        \draw[#3, x=(X),y=(Y)] let \p_=(#1) in (1,1) arc (90:-90:-#2 and 1);
    }

    \fill[red] (Ax0) circle[radius=2pt];
    \node[below] at (Ax0) {$c$};

    \draw[dashed] (A) -- (Ay) node[left] {$f(c)$};

    % front
    \arcflip A 0.2 black
    \arcflip B 0.2 black
    \arcflip A -0.2 dashed
    \arcflip B -0.2 dashed

\end{tikzpicture}
\caption{A disk volume piece of radius $f(c)$ and width $\Delta x$. By radial symmetry, the \textcolor{red}{center of mass} is in the middle of the disk, at point $(c,0)$ as $\Delta x$ tends toward zero.}
\end{figure*}

Using the disk method, we sum the pieces of volume $\pi {[f(x)]}^2 \Delta x$.

For each disk, the centroid simplify lies at the $x$ coordinate, as $\Delta x$ tends to zero, and the disk is symmetrical by rotation.

So, each centroid volume product is given as, \[\pi x {[f(x)]}^2 \Delta x.\]

Then, we partition the interval $[a,b]$ and sum of all of the pieces from $x_0$ to $x_n$, in order to reveal the integral
\[\sum\limits_{i=0}^{n} \pi x_i {[f(x_i)]}^2 \Delta x_i = \int_{a}^{b} \pi x {[f(x)]}^2\,dx.\]

\subsubsection*{2}
Now revolve $\Omega$ around the y-axis and let $S$ be the resulting solid. By symmetry, the centroid of $S$ lies on the $y$-axis and is determined solely by its $y$-coordinate $\overline{y}$.

Show that $\overline{y}V=\int_{a}^{b} \pi x{[f(x)]}^2dx$ where $V$ is the volume of $S$.

\begin{figure*}[ht]
\centering
\begin{tikzpicture}
    \draw[gray, ->] (-5,0) -- (5,0) node[right] {$x$};
    \draw[gray, ->] (0,-1) -- (0,3) node[above] {$y$};

    \draw[blue] (1,2) to[bend right=40] 
        coordinate[pos=0.4] (A)
        coordinate[pos=0.45] (B) (4,2.5);

    \draw (1,2) -- (1,0) node[below] {$a$};
    \draw (4,2.5) -- (4,0) node[below] {$b$};

    \path let \p_=(A) in 
        coordinate (Ax0) at (\x_,0)
        coordinate (Ax1) at ({\x_},-0.5) 
        coordinate (Ay) at (0,\y_)
        coordinate (Ay2) at (0,{\y_/2});


    \draw[dashed] (Ax0) -- (Ax1) node[below] {$c$};
    \node[left] at (Ay) {$f(c)$};

    % #1: point
    % #2: bulge scale
    \def\arcflip #1 #2 #3 {
        \path let \p_=(#1) in coordinate (X) at (\x_,0) coordinate (Y) at (0,\y_); % can't use let binding with coordinate
        \draw[x=(X),y=(Y)] let \p_=(#1) in (1,\y_) arc (0:180:#2 and #3);
    }

    % couldn't get the conditional flipping
    \def\arcflipbase #1 #2 #3 #4 {
        \path let \p_=(#1) in coordinate (X) at (\x_,0) coordinate (Y) at (0,\y_); % can't use let binding with coordinate
        \draw[#4, x=(X),y=(Y)] let \p_=(#1) in (1,0) arc (0:180:#2 and #3);
    }

    \def\flipvertical #1 {
        \draw let \p_=(#1) in (\x_,\y_) -- (\x_,0);
        \draw let \p_=(#1) in (-\x_,\y_) -- (-\x_,0);
    }

    \flipvertical A
    \flipvertical B

    % \def\bendflip #1 #2 { % #1 is shadowed as a point & a shape object.
    %     \draw let \p#1=(#1) in ({-\x#1},\y#1) to[bend right=#2] (#1);
    % }

    % bottom of cylinder
    \arcflipbase A 1 0.4 dashed
    \arcflipbase A 1 -0.4 black
    \arcflipbase B 1 0.5 dashed
    \arcflipbase B 1 -0.5 black

    % midpoint center of mass
    \draw[dashed] (Ay) -- (A);
    \fill[red] (Ay2) circle[radius=2pt]; 

    % top of cylinder
    \arcflip A 1 0.4
    \arcflip A 1 -0.4
    \arcflip B 1 0.5
    \arcflip B 1 -0.5
\end{tikzpicture}
\caption{A cylindrical shell volume piece of width $\Delta x$ and height $f(c)$. By radial symmetry, the \textcolor{red}{center of mass} occurs in the middle of the shell, at a height of $\frac{f(c)}{2}$.}
\end{figure*}

Using the shell method, each cylindrical shell contributes a volume of the difference between two cylinders, 
\[f(x)\pi{(x+\Delta x)}^2-f(x)\pi x^2.\]

We combine this with the fact that the center of mass of each shell occurs at the average value of the functions $f(x)$ and $0$, so $\frac{f(x)}{2}$.

Then, we simplify the center of mass and volume for each piece, noting that a second degree differential is approximated as zero,
\begin{align*}
    &\frac{f(x)}{2}\left[ f(x)\pi{(x+\Delta x)}^2-f(x)\pi x^2 \right] \\
    =& \frac{f(x)}{2} \cdot \pi f(x) \left[ x^2+2x\Delta x+ {(\Delta x)}^2 -x^2 \right] \\
    =& \frac{\pi}{2} {[f(x)]}^2 (2x\Delta x) \\
    =& \pi x {[f(x)]}^2 \Delta x \\
\end{align*}

As in 1, we recognize this to be the integral,
\[\sum\limits_{i=0}^{n} \pi x_i {[f(x_i)]}^2 \Delta x_i = \int_{a}^{b} \pi x {[f(x)]}^2\,dx.\]

\subsubsection*{3(d)}
Use the results in Problems 1 and 2 to locate the centroid of the solid generated by revolving the region below the graph of $f(x)=\sqrt{x}$, $x\in[0,1]$,

(i) about the $x$-axis;

(ii) about the $y$-axis.


For (i), by Problem 1, the centroid about the $x$-axis is given by,
\[\overline{x} = \frac{1}{V} \int_{0}^{1} \pi x^2\,dx.\]

First, we compute the volume of the solid of rotation about the $x$-axis using the disk method,
\begin{align*}
    V&=\int_{0}^{1}\pi{(\sqrt{x})}^2\,dx\\
    &=\pi\left[ \frac{x^2}{2} \right] \Bigg\vert_{0}^{1} \\
    &= \frac{\pi}{2}. \\
\end{align*}

So,
\begin{align*}
    \overline{x} &= \frac{2}{\pi} \int_{0}^{1} \pi x^2\,dx \\
    &= 2 \left[ \frac{x^3}{3} \right] \Bigg\vert_{0}^{1} \\
    &= \frac{2}{3}.
\end{align*}

Thus, \[\overline{x}=\frac{2}{3}.\]

For (ii), by Problem 2, the centroid about the $y$-axis is given by,
\[\overline{y}=\frac{1}{V}\int_{0}^{1} \pi x^2\,dx.\]

We compute the volume of the solid of revolution about the $y$-axis using the shell method,
\begin{align*}
    V&=\int_{0}^{1} 2\pi x \cdot \sqrt{x}\,dx \\
    &= 2\pi \int_{0}^{1} x^{\frac{3}{2}}\,dx \\
    &= 2\pi \left[ \frac{2}{5} x^{\frac{5}{2}} \right] \Bigg\vert_{0}^{1} \\
    &= 2\pi \left( \frac{2}{5} \right) \\
    &= \frac{4\pi}{5}. \\
\end{align*}

Then,
\begin{align*}
    \overline{y}&=\frac{5}{4\pi} \int_{0}^{1} \pi x^2\,dx \\
    &= \frac{5}{4} \left[ \frac{x^3}{3} \right] \Bigg\vert_{0}^{1} \\
    &= \frac{5}{12}.
\end{align*}

So, \[\overline{y}=\frac{5}{12}.\]

\end{document}