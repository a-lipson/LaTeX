\documentclass[../hw2]{subfiles}
\begin{document}
\begin{problem}
Let $u: \D \to \R$. Suppose $u\in \C$ and $\Delta u(x,y)=0$ for all $(x,y)\in \D$.
\begin{enumerate}[label=(\alph*)]
	\item Prove that there exists a holomorphic function on the unit disk such that $\re(f)=u$.
	      Show that the imaginary part of $f$ is defined up to an additive real constant.
	      % Hint: f'(z)=2 \partial u / \partial z. Let g(z)=2 \partial u / \partial z, 
	      % and prove that g is holomorphic. 
	      % Why can we find a primitive F of g such that F' = g?- 
	      % Prove that Re(F) differs from u by a real constant. 
	\item Deduce the Poisson integral representation formula from the Cauchy integral formula.
	      If $u$ is harmonic in the unit disk and continuous on its closure, then, with $z=re^{i\theta}$, \[
		      u(z)=\frac{1}{2\pi}\int_{0}^{2\pi} P_r(\theta-\varphi)u(e^{i\varphi}) \,d\varphi
	      \] where $P_r(\gamma)$ is the Poisson kernel for the unit disk given by  \[
		      P_r(\gamma) = \frac{1-r^2}{1-2r\cos{\theta}+r^2}
		      .\]
\end{enumerate}
\end{problem}
\begin{proof}[Proof of (a)]
	Let $f'=g=2 \partial_z u$.
	We will show that $\partial_{\overline{z}} g = 0$, which implies that $g$ is holomorphic.

	We have the following relation between the mixed Wirtinger derivatives and the Laplacian: \[
		\partial_{\overline{z}}\partial_z = \frac{1}{4}\Delta
		.\]
	Note that the order of the derivatives may be switched.
	So, \[
		\partial_{\overline{z}} g = \partial_{\overline{z}}(2 \partial_z u) =  \frac{1}{2}\Delta u = 0
		.\]
	Hence $g$ is holomorphic on $\D$, and it has a primitive $G$ on  $\D$ such that $G' = g$.

	Then, with $G' = g = 2u'$, integrating both sides gives, \[
		G = 2 \int \frac{d}{dz} u \,dz = 2u + (a+bi)
	\] where $a+bi$ is a complex constant of integration.
	So, \[
		\re(G)=2u+a \text{ and } \im(G) = b
		.\]
	Thus, let $f=\frac{1}{2}G$ and we have found $f$ which satisfies the required conditions.
\end{proof}
\begin{proof}[Proof of (b)]
	% Note that we will show the result up to $u(e^{i\varphi})$ in place of $u(\varphi)$, as the latter representation is an equivalent notation.
	We are given that, on the unit disk, \[
		f(z)=\frac{1}{2\pi}\int_{0}^{2\pi} f(e^{i\varphi}) P_z(\varphi) \,d\varphi
		.\]
	Since $\cos{\gamma}$ is even, then with the other given definition of $P_r(\gamma)$,  \[
		P_{\rho}(-\gamma)=P_{\rho}(\gamma)= \re \left( \frac{ e^{i\gamma} + \rho }{e^{i\gamma} - \rho} \right)
		.\]

	Then, substituting $z=re^{i\theta}$,
	\begin{align*}
		f(re^{i\theta}) & = \frac{1}{2\pi}\int_{0}^{2\pi} f(e^{i\varphi}) \re \left( \frac{e^{i\varphi} + re^{i\theta}}{e^{i\varphi} - re^{i\theta}} \right) \,d\varphi \\
		                & = \frac{1}{2\pi}\int_{0}^{2\pi} f(e^{i\varphi}) \re \left( \frac{e^{i(\varphi-\theta)} + r}{e^{i(\varphi-\theta)}-r} \right) \,d\varphi       \\
		                & =  \frac{1}{2\pi}\int_{0}^{2\pi} f(e^{i\varphi}) P_r(\varphi-\theta) \,d\varphi
		.\end{align*}

	Finally, taking the real part of $f$ and considering that $P_r(\gamma)$ is even, \[
		\re(f(z))=u(z)=u(re^{i\theta})=\frac{1}{2\pi}\int_{0}^{2\pi} u(e^{i\varphi}) P_r(\theta-\varphi)\,d \varphi
		,\] as desired.
\end{proof}
\end{document}
\end{document}
