\documentclass[../hw2]{subfiles}
\begin{document}
\begin{problem}
Analytic functions on the unit disk that cannot be extended analytically past the unit circle.
\begin{definition}[regular]
	Let $f$ be defined on the unit disk  $\D$ with boundary circle $C$.
	A point $w$ on $C$ is regular for $f$ if there is an open neighborhood $U$ of $w$
	and an analytic function $g$ on $U$ such that $f=g$ on $\D\cap U$.
\end{definition}
\begin{lemma}
	A function $f$ defined on $\D$ cannot be continued analytically past the unit circle if no point of $C$ is regular for $f$.
\end{lemma}
Let  \[
	f(z)=\sum_{n=0}^{\infty} z^{2^{n}}, \quad \forall |z|<1
	.\]
Note that the radius of convergence is 1.
Show that $f$ cannot be analytically continued past the unit disk.
% Hint Suppose θ = 2πp/2^k, where p and k are positive integers.
% Let z = re^{iθ}; then |f(re^iθ)|→∞ as r → 1.
\end{problem}
\begin{proof}
	\let\epsilon\varepsilon
	Let $\theta=\frac{2\pi p}{2^k}$ for $p,k \in \Z^+$.
	Let $z=re^{i\theta}$ such that \[
		z^{2^n} = \left( re^{i\theta} \right)^{2^n} = r\exp{2\pi i p (2^{n-k})}
		.\]
	For all $n\ge k$, the above is $\exp{2\pi i p (2^{n-k})}=1$.
	So, we can write \[
		\sum_{0}^{\infty} z^{2^n} = \sum_{0}^{n-1} r\exp{2\pi i p (2^{n-k})} + \sum_{n}^{\infty} r
		.\]
	But, \[
		\lim_{r \to 1^-} \sum_{n}^{\infty} r = \infty \implies \lim_{|z| \to 1} \left| \sum_{0}^{\infty} z^{2^n} \right| = \infty
		.\]
	Therefore $f$ has a singularity on the unit circle where  $\theta=2\pi \frac{p}{2^k}$.

	However, we can use $\frac{p}{2^k}$ for positive integers $p$ and $k$ to produce the binary decimal representation of any number $\frac{\varphi}{2\pi} \in  [0,1] \subset \R$.

	Hence, $\forall \epsilon>0, \exists p, k $ such that \[
		\left|  \frac{\varphi}{2\pi}- \frac{p}{2^k}\right| < \frac{\epsilon}{2\pi}
		\implies |\varphi-\theta| < \epsilon
		.\]
	Therefore $e^{i\theta}$ with the above assignment of $\theta$ will cover all angles in around the unit circle $C$.

	Thus, there exist singularities at all points around the unit circle, so no points are regular on $C$, which means that $f$ cannot be analytically continued past $C$.
\end{proof}
\end{document}
