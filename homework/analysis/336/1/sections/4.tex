\documentclass[../hw1]{subfiles}
\begin{document}
\begin{problem}
Consider the function defined by \[
	f(x+iy)=\sqrt{|x||y|},\quad \text{whenever } x,y \in \R
	.\]
Show that $f$ satisfies the Cauchy-Reimann equations at the origin, yet $f$ is not holomorphic at 0.
\end{problem}
\begin{proof}
	Let $f=u+iv$.
	Since $f=\sqrt{|x| |y|}\in \R$, then $f$ has no imaginary component.
	So, $f=u$ and $f$ vanishes at the origin.

	Hence, $v=0 \implies \partial_x v = \partial_y v = 0 $.
	Then, for the real component, we have that,  \[
		\partial_x f  = \frac{1}{2}\sqrt{\left| \frac{y}{x} \right| } ,\qquad
		\partial_y f = \frac{1}{2}\sqrt{\left| \frac{x}{y} \right| }
		.\]

	We will consider the limit definition of the derivative along the real and imaginary axes: \[
		\lim_{x \to 0} \frac{f(x,0)-f(0,0)}{x} = \lim_{x \to 0} \frac{\sqrt{|x||0|}-0}{x} = 0, \qquad
		\lim_{y \to 0} \frac{f(0,y)-f(0,0)}{y} = \lim_{y \to 0} \frac{\sqrt{|0||y|}-0}{y} = 0
		.\]

	So, indeed, approaching the origin along the coordinate axes, the derivative of $f=u$ vanishes, so the Cauchy-Reimann conditions are trivially satisfied there.

	But, for the path $x=y$, parametrized in $h>0$, \[
		\lim_{h \to 0} \frac{f(h,h)-f(0,0)}{h} = \lim_{h \to 0} \frac{\sqrt{|h||h|}}{h} = \lim_{h \to 0} \frac{|h|}{h} = 1
		.\]

	So, the derivative of $f$ is not continuous at the origin, and hence $f\not\in C^1$ there.
\end{proof}
\end{document}
