\documentclass[../hw1]{subfiles}
\begin{document}
\let\epsilon\varepsilon
\begin{problem}
In this problem we will go through a proof of the Fundamental Theorem of Algebra,
that is: If \[
	p(z)=a_n z^n + \cdots + a_0
\] is a polynomial with an $a_n\neq 0$, then there exists $z_0\in \C$ such that $p(z_0)=0$.
\begin{enumerate}[label=(\roman*)]
	\item Suppose for the sake of contradiction that $p(z) \neq 0$ for all $z \in \C$. Show that the function $g(z) = |p(z)|$ has a minimum at some point $z_0\in \C$. (Hint: Remember that $\C$ is definitely not compact!)
	\item Consider the function $q(z)=\frac{1}{|p(z_0)|}p(z+z_0)$.
	      Show that $q$ is a polynomial with $|q(0)| = 1$ and that $|q(z)|$ has its minimum at $z = 0$.
	\item Show that for any sufficiently small $\epsilon>0$, there is some $\theta$ for which $|q(\epsilon e^{i\theta})|<1$, which provides the desired contradiction.
\end{enumerate}
\end{problem}
\begin{proof}[Proof of (a)]
	By the triangle inequality, we have that \[
		|p(z)| = \left| \sum_{0}^{\infty} a_n z^n \right| \le \sum_{0}^{\infty} | a_n z^n |
		.\]
	So $g(z)=|p(x)| = O(|z|^n)$.

	Then, as $|z|\to \infty$, we have that $g\to \infty$;
	i.e., for sufficiently large $R$, we have \[
		R < |z| \implies a_0=|p(0)| < |p(z)| = g \tag{*}
		.\]

	Consider the closed disk $D$ of radius $R$ centered at the origin.
	Since $D$ is closed and bounded, then it is compact.

	Since $p$ is continuous, then $g=|p|$ is also continuous.

	Then, by EVT, $g$ attains a minimum value on $D$ at some point  $z_0\in D$.

	So, $|p(z_0)|$ is the minimum value of $g$ on  $D$.

	For all $z$ outside $D$, then $R < |z|$, so we have $|p(0)| < g$ by (*).

	If $|p(z_0)| < |p(0)|$, then $g$ attains a global min at $z_0$.\\
	If $|p(x_0)| = |p(0)|$, then $g$ attains a global min at either $z_0$ or 0.\\
	In either case, $g$ attains its min at some point in $\C$.
\end{proof}
\begin{proof}[Proof of (b)]
	Since $p(z)$ is a polynomial, then  $p(z+z_0)$ is also a polynomial, just translated by $z_0$.
	Then, $\frac{1}{|p(z_0)|}$ is a constant.
	So, $q$, is a scaled and translated polynomial, which is still a polynomial.

	Note that  \[
		|q(0)|=\frac{|p(0+z_0)|}{|p(z_0)|}=1
		.\]

	Since $|p(z_0)|$ is the min of $|p(z)|$, then  $|p(z_0)|\le |p(z+z_0)|$ for all $z\in \C$.

	Hence, $1\le |q(z)|$ for all $z\in \C$, with equality when $z=0$.
	Thus, $|q(z)|$ has a min at $z=0$.
\end{proof}
\begin{proof}[Proof of (c)]
	Since $q$ is a polynomial with  $q(0)=1$, then it can be represented with a finite series,  \[
		q(x)=1+\sum_{1}^{n} c_k z^k, \qquad q(\varepsilon e^{i\theta}) = 1 + \sum_{1}^{n} c_k \epsilon^k e^{i k \theta}
		.\]

	Note that, for $\epsilon$ sufficiently small, \[
		|c_k \epsilon^k | > |c_{k+1} \epsilon^{k+1}| \tag{*}
	\] regardless of the constants $c_k,c_{k+1}$.

	Assume that $c_k$ is the lowest indexed nonzero coefficient.
	Then, \[
		q(\epsilon e^{i\theta}) = 1 + c_k \epsilon^k e^{i k \theta} + \psi(\theta)
		,\] where $|\psi(\theta)| < | c_k \epsilon^k e^{i k \theta}|$ by (*).

	We wish to choose $\theta$ in the opposite direction of  $c_k$.
	Since $c_k=|c_k|e^{i\varphi}$, then we will consider $\theta=\frac{\pi-\varphi}{k}$ so that \[
		{i k \theta} = {i k \left( \frac{\pi-\varphi}{k} \right) } = {i(\pi-\varphi)}
		.\]
	Therefore, \[
		c_k e^{i k \theta} = |c_k|e^{i\varphi}e^{i(\pi-\varphi)}=|c_k|e^{i\pi}=-|c_k|
		.\]

	So, for this choice of $\theta$,  \[
		q(\epsilon e^{i \theta}) = 1 - |c_k| \epsilon^k + \psi(\theta)
		,\] and $|\psi(\theta)| < |c_k|\epsilon^k$.

	Thus, $q(\epsilon e^{i\theta})<1$, a contradiction.

	So, there must exist $z$ such that  $p(z)=0$, meaning that all polynomials in  $\C$ must have at least one root.
\end{proof}
\end{document}
