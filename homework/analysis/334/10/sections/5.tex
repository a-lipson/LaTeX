\documentclass[../hw10]{subfiles}
\begin{document}
\begin{problem}[5]
Let $\mathbf{G} = (-yz^2,xz^2,\exp{(-z^2)})$ and $\mathbf{F}=\text{curl}\,\mathbf{G}$.
Let $S$ be the surface defined in cylindrical coordinates by  $r=1+z^2, r\le 3$ oriented with normals pointing towards the $z$-axis.
Divide  $S$ into the portion  $S_+$ in positive  $z$, and  $S_-$ in negative  $z$.

Determine whether the flux of  $\mathbf{F}$ through $S_+$ or  $S_-$ is larger.
\end{problem}
\begin{proof}
	By stoke theorem, we have \[
		\iint_S \mathbf{F}\cdot d\mathbf{S} = \iint_S \text{curl}\,\mathbf{G}\cdot d\mathbf{S} = \oint_C \mathbf{G}\cdot d\mathbf{x}
	\] where $\partial{S}=C$ and $C$ is positively oriented.

	So, we will consider the boundary curves of $S_+$ and $S_-$, which we will call $C_+$ and  $C_-$ respectively.

	We will set up a difference equation for the upper and lower surfaces, \[
		\iint_{S_+}\mathbf{F}\cdot d\mathbf{S}-\iint_{S_-}\mathbf{F}\cdot d\mathbf{S}
		= \oint_{C_+}\mathbf{G}\cdot d\mathbf{x}-\oint_{C_-}\mathbf{G}\cdot d\mathbf{x}
		.\]

	Since the image of $S$ in the $xy$-plane is a unit circle ($r=1$ where $z=0$), then the part of the boundaries $C_+$ and $C_-$ on the unit circle will be traversed in opposite directions.
	Therefore the line integrals over these paths will cancel each other in the difference equation.

	So, we are left with the parts of $C_+$ and  $C_-$ that bound $r=3$ at $z= \pm \sqrt{2} $ respectively.
	We will call these curves $\gamma_+$ and  $\gamma_-$.

	We will begin by parametrizing $C_+$ with $\gamma_+(t)=(3\cos{t},3\sin{t},\sqrt{2})$.
	Since the normals of $S$ point towards the  $z$-axis, to maintain continuity of the normals, we must have an anticlockwise orientation for  $\gamma_+$.
	So, $t\in [0,2\pi]$.
	Then, $\gamma'_+(t)=(-3\sin{t},3\cos{t},0)$ and
	$\mathbf{G}(\gamma_+(t))=(-6\sin{t},6\cos{t},e^{-2})$.

	So, \[
		\oint_{\gamma_+}\mathbf{G}\cdot d\mathbf{x}
		= \int_{0}^{2\pi} \mathbf{G}(\gamma_+(t))\cdot \gamma'_+(t) \,dt
		= \int_{0}^{2\pi} 18 \,dt
		= 36\pi
		.\]

	For $\gamma_-$, we will perform the same, except the  $z$ coordinate of the parametrization will be  $-\sqrt{z}$, and the direction will be reversed.
	This will not impact the integrand, but rather the bounds for the integral, which will be reversed.
	So, the line integral around $\gamma_-$ will be  $-36\pi$.

	Thus, the difference equation is will be positive, indicating that the flux on $\mathbf{F}$ through $S_+$ is larger than that over $S_-$.
\end{proof}
\end{document}
