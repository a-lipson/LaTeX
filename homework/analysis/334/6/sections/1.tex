\documentclass[../hw6]{subfiles}
\begin{document}
We will prove
\begin{theorem}
	Suppose $f:[a,b]\longrightarrow\R$ bounded, continuous on $[a,b]\setminus D$, where $D$ has (Lebesgue) measure zero, then $f$ is (Reimann) itegrable.
\end{theorem}
\begin{definition}
	The \textit{oscillation} of $f$ at $x$ is  \[
		\text{osc}(f,x)=\lim_{\delta \to 0^+} \sup \{|f(y)-f(z)|\ \mid\ y,z\in B_{\delta}(x)\}
		.\]
\end{definition}
Let $D_s = \{x \in [a,b] \ \mid\ \text{osc}(f,x)\ge s\} $.\\
Let $D = \{x \in [a,b]\ \mid\ f \text{ discontinuous at } x \} $.\\
Let $m=\inf f$ and  $M = \sup f$.

\begin{problem}[1]
Show that if $S$ has measure zero, then any subset of $S$ also has measure zero.
\end{problem}
\begin{proof}
	Since $S$ has measure zero, then  $\forall \delta>0$, $S$ has a cover $C\subset \bigcup\limits_{i=1}^{\infty} B_{r_i}(x_i)$ with $\sum\limits_{i=1}^{\infty} r_i < \delta$.
	Then, any subset can use the same cover, giving such a subset measure zero as well.
\end{proof}
\end{document}
