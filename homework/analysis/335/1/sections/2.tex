\documentclass{subfiles}
\begin{document}
\begin{problem}[2]
Determine whether the following integrals converge.
\begin{enumerate}[label=\roman*)]
	\item $\int_{\frac{\pi}{2}}^{\pi} \cot{x} \,dx$

	      Note that there is a singularity of cotangent at $\pi$.
	      Around $x=\pi$, $\sin{x}\approx x - \pi$ and $\cos{x} \approx -1$.

	      So, $\cot{x}\approx \frac{1}{\pi - x}$, which means that, near $x = \pi,\, \cot{x}\approx O(\frac{1}{x})$, which diverges.

	      Thus, (i) diverges by comparison.

	\item $\int_{0}^{1} \frac{dx}{x^{1 / 2}{(x^2 + x)}^{1 / 3}}$

	      We see that (ii) can also be expressed as $\int_{0}^{1} \frac{dx}{x^{5 / 6}{(x + 1)}^{1 / 3}}$.

	      Let $g(x)=\frac{1}{x^{5 / 6}}$. $g$ converges on $x \in [0,1]$.

	      Then, $\lim_{x \to \infty} \frac{x^{5 / 6}}{x^{5 / 6}{(x + 1)}^{1 / 3}} = 0$.

	      Since $\lim_{x \to \infty} \frac{f(x)}{g(x)} = 0$ and $g$ converges on the given interval,
	      then, by the comparison test, (ii) converges as well.

	\item $\int_{0}^{1} \frac{1-\cos{x}}{\sin^3{2x}} \,dx$

	      Note that there is a singularity at zero,
	      and this is the only such value given that the next root of sine occurs at $\frac{\pi}{2}>1 $.

	      Near $x=0$,  $\sin{2x}\approx 2x$ and $1-\cos{x}\approx \frac{x^2}{2}$.

	      So, the integrand is approximately  $\frac{\frac{x^2}{2}}{{(2x)}^3} = \frac{1}{16x}$, which diverges at zero.

	      So, (iii) diverges by comparison.
\end{enumerate}
\end{problem}
\end{document}
