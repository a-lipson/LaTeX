\documentclass[../hw7]{subfiles}
\begin{document}
\begin{problem}
Consider the ring of integers Z.
\begin{enumerate}[label=\alph*)]
	\item  Show that for each nonnegative integer $n$, $(n)$ is an ideal of $\Z$, and that all these ideals are distinct.
	\item Prove that these are all the ideals of $\Z$.
\end{enumerate}
% Hint: Show that if I is a nonzero ideal of Z, then I must contain a smallest
% positive integer c. Then show that I = (c).
\end{problem}
\begin{proof}[Proof of a]
	First, $(n)$ is non empty for all nonnegative integers $n$.

	Second $(n)$ is closed under subtraction.
	For $a, b \in  (n)$, $k_1,k_2\in \Z$, \[
		a+b = nk_1-nk_2=n(k_1-k_2)\in (n)
		.\]

	Third, $(n)$ is closed under multiplication with  $\Z$.
	For $a = nk \in (n)$ and $r,k\in \Z$, then \[
		ra = n(rk)\in (n)
		.\]

	So $(n)$ is an ideal in  $\Z$ for any nonnegative integer.
\end{proof}
\begin{proof}[Proof of b]
	Let $I$  be a nonzero ideal of $\Z$.
	Since $I$  is nonzero, it contains at least one nonzero element.

	Since $I$ contains nonzero elements, it must contain some positive integers (if $x \in I$ is negative, then $-x \in I$ is positive because ideals are closed under multiplication by -1).

	Let $c$  be the smallest positive integer in $I$.
	%such a smallest $c$ must exist since integers are ordered.
	%This c exists by the well-ordering principle for positive integers.

	First, $(c) \subset I$.
	Since $c\in I$, and $I$  is an ideal, then $c\cdot k\in I \forall k\in \Z$.
	Since all multiples of $c$ are in $I$, then  $(c)\subset I$.

	Next, $I\subset (c)$.
	For any $a\in I$, we have, by the division algorithm, $a=cq+r$ where $0\le r\le c$.

	So, $r=a-cq$.
	Since  $a,c\in I$, which is an ideal,
	then $r=a-cq\in I$.

	But $c$ is the smallest positive integer in  $I$,
	so if  $r>0$, then this contradicts the minimality of  $c$.

	Therefore $r=0$ and $a=cq \in (c)$, which implies that $I \subset (c)$.

	Thus $I=(c)$, so every nonzero ideal of $\Z$ is of the form $(c)$ for some positive integer $c$.

	(0) is also included as (0) = {0}.

	Thus, all ideals of $\Z$  are of the form $(n)$ for some nonnegative integer $n$ .
\end{proof}
\end{document}
