\documentclass[../hw3]{subfiles}
\begin{document}
\begin{problem}
Prove Theorem 3.1. \\
Let $R,S$ be rings.
\begin{enumerate}[label=\roman*)]
	\item Define addition and multiplication on the Cartesian product  $R\times S$ by
	      \begin{align*}
		      (r,s)+(r',s')      & = (r + r', s + s')         \\
		      (r,s)\cdot (r',s') & = (r \cdot r', s \cdot s')
		      .\end{align*}
	      Then $R\times S$ is a ring.

	\item  $R,S$ commutative  $\implies$ $R\times S$ commutative.
	\item  $R,S$ with identity  $\implies$ $R\times S$ with identity.
\end{enumerate}\end{problem}
\begin{proof}[Proof of i]
	The closure and inverse of addition are retained from the properties of addition in the rings $R$ and $S$.

	We add only elements from $R$ to other such elements and the same with $S$.
	Explicitly, we could also write \[
		(r,s) {\bigcirc\mkern-16mu+}_{R\times S}(r',s') = (r {\bigcirc\mkern-16mu+}_R r', s {\bigcirc\mkern-16mu+}_S s')
		.\]

	The zero element of  $R\times S$ is given by $(0_R,0_S)$ where  \[
		(r,s)+(0_R,0_S)=(r+0_R,s+0_S)=(r,s)
		.\]

	Similarly, the closure under multiplication is retained as well.
\end{proof}
\begin{proof}[Proof of ii]
	$R,S$ commutative implies that $rr'=r'r$ and $ss' = s's$ .
	Then, \[
		(r,s)(r',s')=(rr',ss')=(r'r,s's)=(r',s')(r,s)
		.\]
\end{proof}
\begin{proof}[Proof of iii]
	$R,S$ with identity gives that there are both $1_R$ and  $1_S$.
	Then,  $1_{R\times S} = (1_R,1_S)$.
	We see that, for left multiplication, \[
		(r,s)(1_R,1_S)=(r\cdot 1_R,s\cdot 1_S)=(r,s)
		.\]
	The same holds for right multiplication given that $1\cdot a = a = a\cdot 1$.

\end{proof}
\end{document}
