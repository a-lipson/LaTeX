\documentclass[../hw3]{subfiles}
\begin{document}
\begin{problem}
Which of the following are subrings of $M(\R)$ and which have identity?
\end{problem}
\begin{enumerate}[label=\alph*)]
	\item $\begin{pmatrix} 0 & r \\ 0 & 0 \end{pmatrix} ,\, r\in \Q$.

	      This is a subset of $M(\R)$ and contains the zero matrix. \\
	      This set is closed under addition as $\Q$ is closed under addition. \\
	      Similarly, this set has an additive inverse given that $-r\in \Q$ as well. \\
	      This set is closed under multiplication because multiplication maps only to the zero matrix.
	      This set does not have an identity because all multiplication maps to zero, so there cannot be an element 1 such that $\forall a,\, 1\cdot a = a$ because we already have $1\cdot a=0$.
	      Since this the ring axioms hold for this set, then this set forms a subring of $M(\R)$.

	\item $\begin{pmatrix} a & b \\ 0 & c \end{pmatrix},\, a,b,c\in \Z$.

	      The addition properties of closure, an inverse, and a zero, are satisfied as in $\Z$. \\
	      Furthermore, we see that the multiplication of two elements from this set produce only a zero in the bottom left entry of the product matrix, and a linear combination of other elements of $\Z$ elsewhere. Thus the set is closed under multiplication because $\Z$ is closed under linear combinations and the bottom left entry remains zero. \\
	      This set contains an identity element which is the $2\times 2$ matrix identity $I$.
	      Since the ring axioms and identity hold for this set, then it forms a subring with identity.

	\item $\begin{pmatrix} a & b \\ c & 0 \end{pmatrix},\, a,b,c\in \R$.

	      This set is not closed under multiplication. Consider the bottom right entry of a product matrix, it can be nonzero and therefore an element not belonging to this set. \\
	      So, this set does not form a subring.

	\item $\begin{pmatrix} a & 0 \\ a & 0 \end{pmatrix} ,\, a\in \R$.

	      This set is closed under addition and has both a zero and an additive inverse given as in $\R$. \\
	      Consider multiplication for $a,b\in \R$, \[
		      \begin{pmatrix} a & 0 \\ a & 0 \end{pmatrix} \begin{pmatrix} b & 0 \\ b & 0 \end{pmatrix} = \begin{pmatrix} ab & 0 \\ ab & 0 \end{pmatrix}
		      .\] We see that multiplication is closed as it is for scalars in $\R$, and also that we have the identity element $\begin{pmatrix} 1 & 0 \\ 1 & 0 \end{pmatrix} $.
	      Since this set conforms to the axioms of a ring and also has identity, then it is a subring with identity.

	\item $\begin{pmatrix} a & 0 \\ 0 & a \end{pmatrix},\, a\in \R$.

	      We see that this set is just a scalar matrix of the form $aI$ where  $I$ is the $2\times 2$ identity matrix. \\
	      So, this set is closed under addition and multiplication and there exists both additive and multiplicative inverses and identities given that the same holds for scalars $a\in \R$. \\
	      Thus, this set forms a subfield of $M(\R)$, and therefore a subring with identity as well.

	\item $\begin{pmatrix} a & 0 \\ 0 & 0 \end{pmatrix} ,\, a\in \R$.

	      This set is closed under addition, has a zero, and has an additive inverse, given as in $\R$. \\
	      This set is closed under multiplication as the only entry in a product matrix will be the top left and $\R$ is closed under multiplication. \\
	      The identity element is given by $\begin{pmatrix} 1 & 0 \\ 0 & 0 \end{pmatrix} $. \\
	      Since this set conforms to the ring axioms and has an identity, then it is a subring with identity.

\end{enumerate}
\end{document}
