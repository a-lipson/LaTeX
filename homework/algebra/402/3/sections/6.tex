\documentclass[../hw3]{subfiles}
\begin{document}
\begin{problem}
Let $d\in \Z$ be not a perfect square.
Show $\Q(\sqrt{d} ) = \{ a + b\sqrt{d}\ \mid\ a,b \in \Q \} $ is a subfield of $\C$.
\end{problem}
\begin{proof}
	Since $d$ is not a perfect square, then  $\sqrt{d}$ cannot be simplified.

	Note that $d<0 \implies \sqrt{d} = i\sqrt{(-d)} \in \C$, so $\Q(\sqrt{d})$ would contain complex elements.
	If $d\ge 0$, then we still have $\Q(\sqrt{d})\subset \R\subset \C$.
	Thus, $\Q(\sqrt{d})\subset \C$.

	$\Q(\sqrt{d})$ is closed under addition as in $\Q$, \[
		(a+b\sqrt{d})+(a'+b'\sqrt{d})=(a+a')+(b+b')\sqrt{d}
		.\]

	For closure under multiplication, consider
	\begin{align*}
		(a+b\sqrt{d})(a'+b'\sqrt{d}) & = aa'+ab'\sqrt{d}+ba'\sqrt{d}+bb'(\sqrt{d})^2 \\
		                             & = (aa'+bb'd)+(ab'+ba')\sqrt{d}
		.\end{align*}
	Since the product of $d\in \Z$ with elements is $\Q$ belongs to $\Q$ and $\Q$ is closed under addition and multiplication,
	then $\Q(\sqrt{d})$ is closed under multiplication.

	Next, we have the zero element $0+0\sqrt{d}=0$.

	We have the additive inverse of any $a+b\sqrt{d}\in \Q(\sqrt{d})$ as $(-a)+(-b)\sqrt{d})$, where \[
		(a+b\sqrt{d})+((-a)+(-b)\sqrt{d}) = (a+(-a)) + (b+(-b))\sqrt{d} = 0+0\sqrt{d} = 0
		.\]

	We also have the multiplicative identity element $1+0\sqrt{d}$, where $\forall a,b\in \Q$ \[
		(1+0\sqrt{d})(a+b\sqrt{d})=a+b\sqrt{d}
		.\]

	For the multiplicative inverse, we have that $(a+b\sqrt{d})(a'+b'\sqrt{d})=(aa'+bb'd)+(ab'+ba')\sqrt{d}$ from above.

	So, $(aa'+bb'd)+(ab'+ba')\sqrt{d}=1+0\sqrt{d}\implies ab+ba' = 0 \land aa'+bb'd=1$.

	Set $b'=\frac{-ba'}{a}$ so that we have,
	\begin{align*}
		aa'+b \left( \frac{-ba'}{a}\right) d & = 1                                            \\
		a' \left(a- \frac{b^2 d}{a} \right)  & = 1                                            \\
		a'(a^2-b^2 d )                       & = a                                            \\
		a'                                   & = \frac{a}{a^2-b^2 d}                          \\
		b'                                   & = -\frac{b}{a}\left(\frac{a}{a^2-b^2 d}\right) \\
		                                     & = -\frac{b}{a^2-b^2 d}
		.\end{align*}
	So, ${(a+b\sqrt{d} )}^{-1} = \frac{1}{a^2-b^2 d}(a-b\sqrt{d} )$.
	Note that this is very similar to the inverse in $\C$!

	Since $\frac{1}{a^2-b^2 d}(a-b\sqrt{d} )\in \Q(\sqrt{d})$, then every element in $\Q(\sqrt{d})$ has an inverse.

	Since $\Q(\sqrt{d})\subset \C$ satisfies the ring axioms (with the ordinary add. \& mul. ops. as in $\C$) and has both a multiplicative inverse and identity, then it is a subring and subfield of $\C$.
\end{proof}
\end{document}
