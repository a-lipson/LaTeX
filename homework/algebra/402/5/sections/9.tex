\documentclass[../hw5]{subfiles}
\begin{document}
\begin{problem}
Consider $f(x)=x^4-6x^2+1$.
\begin{enumerate}[label=\alph*)]
	\item Write $f$ as a product of irreducible polynomials in  $\Q[x]$.
	\item Write $f$ as a product of irreducible polynomials in  $\R[x]$.
	\item  Explain why parts a and b do not contradict the Unique Factorization Theorem in $F[x]$.
\end{enumerate}
\end{problem}
\begin{proof}[Proof of a]
	$f$ factors as  $(x^2+2x-1)(x^2-2x-1)$.
	These factors themselves are polynomials of degree 2 with irrational roots, so they are irreducible in $\Q$.
	Thus, we're done.
\end{proof}
\begin{proof}[Proof of b]
	$f$ factors further in  $\R[x]$;
	the quadratics in part a factor in $\R[x]$, giving $x= \pm (1 \pm \sqrt{2})$,
	which can be written as the product of irreducible linear polynomials, \[
		f(x)=(x+(1+\sqrt{2} ))(x-(1+\sqrt{2} ))(x+(1-\sqrt{2} ))(x-(1-\sqrt{2} ))
		.\]
\end{proof}
\begin{proof}[Proof of c]
	The Unique Factorization Theorem says that factorizations are unique within a given ring, but $\Q[x]$ and $\R[x]$ are different rings; so, while each ring has its own unique factorization for any given polynomial, these need not be the same between different rings.

	What about isomorphic rings? i would suspect that the factor structure is maintained across isomorphic rings.
\end{proof}
\end{document}
