\documentclass[../hw5]{subfiles}
\begin{document}
\begin{problem}
Use the Rational Root Theorem to write the polynomials as a product of irreducibles in $\Q[x]$.
\begin{enumerate}[label=\roman*)]
	\item $f(x)=3x^5+2x^4-7x^3+2x^2$.
	\item $g(x)=2x^4-5x^3+3x^2+4x-6$.
\end{enumerate}
\end{problem}
\begin{proof}[Proof of i]
	First, we can factor out $x^2$, so we have $f=x^2(3x^3 + 2x^2 - 7x + 2)$.

	Now, we can use the Rational Root Test on the right-hand factor.
	This gives us a root of the form $q=\frac{r}{s} \in \Q$ where $r|2$ and $s|3$.

	So, we could have $r= \pm 1, \pm 2$ and $s= \pm 1, \pm 3$, which gives the possible
	$q= \pm 1, \pm 2, \pm \frac{1}{3} \pm \frac{2}{3}$.

	We will first check $x=1$,  $3(1)^3+2(1)^2-7(1)+2=0$, so $x-1$ is a factor of  $f$.

	Now that we have a factor, we can use it to divide and simply the remaining unfactored polynomial.

	With polynomial long division, we get that $f=x^2(x-1)(3x^2+5x-2)$.

	We can apply the Rational Root Test once more on this right-hand factor.
	Note that we have the same possibilities for the roots as in the previous time when we applied the test as the constant coefficient changed only in sign.

	We will pick another root to check, say $x=-2$;
	\footnote{The Rational Root Test is good for getting started on a high-degree polynomial which we could not otherwise factor, but, at this point, with a quadratic, we already have more efficient methods to factor; hence, we knew that $x=-2$ was a good root to check!}
	then, $3(-2)^2+5(-2)-2=0$, so $x+2$ is a factor of the quadratic.

	Polynomial long division then gives us that, \[
		f(x)=x^2(x-1)(x+2)(3x-1)
		.\]
\end{proof}
\begin{proof}[Proof of ii]
	By the Rational Root Test, a root $q=\frac{r}{s}\in \Q$ must have that $r|6$ and  $s|2$.

	So, $r= \pm 1, \pm 2, \pm 3, \pm 6$ and $s= \pm 1, \pm 2$ give
	$q= \pm 1, \pm 2, \pm 3, \pm 6, \pm \frac{1}{2}, \pm \frac{3}{2}$.

	We will try $x=1$,  $2(1)^4-5(1)^3+3(1)^2+4(1)-6\neq 0$.

	Next, we will try $x=-1$,  $2(-1)^4-5(-1)^3+3(-1)^2+4(-1)-6 =0$, so $1$ is a root of  $g$.

	We will divide $g$ by its factor  $x+1$ and apply RRT again.

	Polynomial long division gives,  $g=(x+1)(2x^3-7x^2+10x-6)$, which contain the same leading and constant coefficients as in the first application of RRT.

	So, we can try $x=\frac{3}{2}$, $2\left( \frac{3}{2} \right)^3-7\left( \frac{3}{2} \right)^2+10\left( \frac{3}{2} \right)-6=\frac{27}{4}-\frac{63}{4}+\frac{60}{4}-\frac{24}{4}=0$, so $x=\frac{3}{2}$ is a root of $g$.

	We can reduce the polynomial once more by long division with the factor $x-\frac{3}{2}$, which gives $g(x)=(x+1)\left(x-\frac{3}{2}\right)(2x^2-4x+4)$.

	We can move a factor of 2 from the right-most term to arrive at, \[
		g(x)=(x+1)(2x-3)(x^2-2x+2)
		.\]
	Note that $x^2-2x+2$ is irreducible in $\Q[x]$ as it has a negative discriminant and therefore no rational roots.

	Thus, the irreducible factors of $g$ are given above.
\end{proof}
\end{document}
