\documentclass[../hw8]{subfiles}
\begin{document}
\begin{problem}
Prove that the principal ideal $(x-1)$ in  $\Z[x]$ is prime but not maximal
\end{problem}
\begin{proof}[Proof of primeness]
	We will to show that $\Z[x] / (x-1)$ is an integral domain and hence the ideal $(x-1)$ is prime.

	Consider  $f(x)g(x)=0$ in $\Z[x] / (x-1)$.
	Then we must have that $x-1 \mid f(x)g(x) $.

	Since $x-1$ is a linear and therefore irreducible polynomial in  $\Z[x]$,
	then we must have that either $f(x)=0$ or  $g(x)=0$ in $\Z[x] / (x-1)$.

	Thus, there are no zero divisors, hence $\Z[x] / (x-1)$  is an integral domain.
\end{proof}
\begin{proof}[Proof of non maximality]
	We will show that there exists an ideal $I$ in $\Z[x]$ such that $(x-1)\subsetneq I\subsetneq \Z[x]$.

	Consider $I=(x-1,2)$.
	Clearly  $(x-1)\subset (x-1,2)$.

	But, $2\in (x-1,2)$ yet $2\not\in (x-1)$, so $(x-1)\subsetneq (x-1,2)$.

	Then, for all $h(x)\in (x-1,2)$, $h(x)=(x-1)f(x)+2g(x)$ for some  $f,g\in \Z[x]$.

	Then, at $x=1$, $h$ must be even.
	Therefore, the constant function $1\not\in (x-1,2)$, but $1 \in \Z[x]$.

	So, $(x-1,2)\subsetneq \Z[x]$.

	Thus, $(x-1)$ is not maximal.
\end{proof}
\end{document}
