\documentclass[../9extra]{subfiles}
\begin{document}
\begin{problem}[2]
Give an example of a shrinking map that is not a contraction map.
\end{problem}
\begin{proposition}[2]
	$\forall \epsilon : 0<\epsilon<1$, the map $ f:[0,1]\longrightarrow \R$ defined as $f(x)=(1-\epsilon)x - \frac{x^2}{2}$ is a shrinking map that is not a contraction map.
\end{proposition}
\begin{proof}[Proof of 2]
	Since a contraction map requires, for some fixed $\alpha \in (0,1)$, that $\forall x,y \in K,\, x \neq y$, \[
		|f(x)-f(y)| < \alpha|x-y|
		,\]
	then we wish to find a map such that this relationship will not hold for any fixed choice of $\alpha$.

	As $x$ approaches zero,  $f'(x)$ can get arbitrarily close to 1, but no fixed  $\alpha$ will work as we can always choose a smaller  $\epsilon$.
\end{proof}
\end{document}
