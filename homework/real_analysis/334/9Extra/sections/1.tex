\documentclass[../9extra]{subfiles}
\begin{document}
\begin{problem}[1]
Let $K\subset \R^{n}$ be compact.
Let $f:K\longrightarrow K$ be a \textit{shrinking map}.
$\forall x,y \in K,\, x \neq y \implies |f(x)-f(y)| < |x-y|$.

Prove that $f$ has a unique fixed point  $x \in K : x = f(x)$.
\end{problem}

If $K$ compact with  $K \supset C_1 \supset C_2 \supset \cdots$ nested sequence of non-empty closed subsets $C_i$, then \[
	\bigcap_{i=1}^{\infty} C_i \neq \O \tag{*}
	.\]

\begin{proposition}[Continuity of the shrinking map]
	$f$ is uniformly continuous.
\end{proposition}
\begin{proof}[Proof of Continuity Proposition]
	Choose $\delta = \epsilon$.
	Then, $\forall \epsilon>0,\, \exists \delta>0$, \[
		|f(x)-f(y)| < |x-y| < \delta = \epsilon
		.\]
	So, $f$ is uniformly continuous as  $\delta$ depends solely on  $\epsilon$.
\end{proof}

% \begin{proposition}[Uniqueness of the fixed point]
% 	The fixed point of $f$ is unique.
% \end{proposition}
% \begin{proof}[Proof of Uniqueness Proposition]
% 	Suppose, for a contradiction, that there were $x,y$ fixed points with  $x \neq y$
% 	and $f(x)=x$,  $f(y)=y$.
%
% 	Then,  $|f(x)-f(y)| = |x-y|$.
% 	But we had, by the definition of  $f$, $|f(x)-f(y)| < |x-y|$, which is a contradiction.
%
% 	Thus, $x=y$ and the fixed point is unique.
% \end{proof}

\begin{proposition}[Successive applications of the map form a nested sequence]
	Let $C_0 = K$, and  $C_{i + 1} = f(C_i)$.
	\begin{enumerate}[label=(\alph*)]
		\item $C_i$ closed.
		\item $C_i$ non-empty.
		\item $C_i$ nested such that $K \supset C_1 \supset C_2 \supset \cdots$
	\end{enumerate}
\end{proposition}

\begin{proof}[Proof of Sequence Proposition (a)]
	By induction, since $f$ continuous and $C_0$ compact, then $\forall i$, $C_i$ is compact as well.
\end{proof}

\begin{proof}[Proof of Sequence Proposition (b)]
	Since $f$ maps from $K$ to $K$ and $C_0 = K$ is non-empty, then each $C_i$, as the image of $f$, must be non-empty.
\end{proof}

\begin{proof}[Proof of Sequence Proposition (c)]
	% Let $S\subset K$. So, \[
	% 	|f(x)-f(y)| < |x-y|
	% 	\implies \underset{x,y \in S}{\sup} |f(x)-f(y)| < \underset{x,y \in S}{\sup} |x-y|
	% 	\implies \text{diam}\,{f(S)} < \text{diam}\,{S}
	% 	.\]
	% Since $f$ maps from  $K$ to $K$,
	% and $\forall S \subset K,\, \text{diam}\,{f(S)} < \text{diam}\,{S}$,
	% then $f(K)\subset K$.
	% So, by induction, $C_{i+1} \subset C_i$.
	By induction, since $f(K)\subset K$ and $C_{i+1} = f(C_i)$, then $C_{i+1}\subset C_i$.
\end{proof}

\begin{proof}[Proof of Problem 1]

	Let $S\subset K$. So, \[
		|f(x)-f(y)| < |x-y|
		\implies \underset{x,y \in S}{\sup} |f(x)-f(y)| < \underset{x,y \in S}{\sup} |x-y|
		\implies \text{diam}\,{f(S)} < \text{diam}\,{S}
		.\]
	Since $f$ maps from  $K$ to $K$,
	and $S, C_i \subset K$, then $ \text{diam}\,{f(S)} < \text{diam}\,{S} \implies \text{diam}\,{C_{i+1}} < \text{diam}\,{C_i}$.

	Let $(x_n)\subset \R$ be the sequence defined by $x_i = \text{diam}\,{C_i}$.
	So, $\lim_{n \to \infty} x_n = \text{diam}\,S$.

	Since $\text{diam}\,{C_{i+1}} < \text{diam}\,{C_i}$, then $(x_n)$ is decreasing.
	Since $\text{diam}\,{C_i}\ge 0$, then $x_n$ is bounded below by zero.
	Since $(x_n)\subset \R$ is decreasing and bounded below, it must converge.

	Suppose, that $(x_n)$ converges to $m>0$.

	Then, $S=\bigcap C_i$ must contain at least two points $x,y$ such that $|x-y| = m$
	But, $\forall i,\, x,y \in C_i$ means that we could not have $\text{diam}\,{C_{i+1}} < \text{diam}\,{C_i}$,
	which is a contraction.
	Thus, $(x_n)$ must converge to zero.

	Since $(x_n)\to 0$, then $\text{diam}\,S = 0$.

	By (*), $\forall i,\, \bigcap_{i=1}^{\infty} C_i \neq \O$. %, the infinite intersection of $C_i$ is non-empty.

	Since $\text{diam}\,{S} = 0$ and $S\neq \O$,
	then $S = \{x\}$.
	Since $x \in C_i \implies f(x) \in C_{i+1}$,
	then $f(x) \in S$.
	So, $f(x) = x$.

	% Let $g : K \longrightarrow \R$ be the distance between $x$ and its image under $f$, \[
	% 	g(x)= |f(x)-x|
	% 	.\]
	% Since $f$ is continuous, then $g$ is also continuous.
	% Since  $K$ is compact and $g$ is continuous and non-negative, then, by EVT, there is a minimum of $g$ at $x \in K$, \[
	% 	0\le \underset{x \in K}{\min{g}} = g(a)
	% 	.\]
	% We wish to show that $g(a)$ must be zero.
	% This $a$ is the fixed point of $f$.
	%
	% First, if $g(a)=0$, then we're done.
	%
	% Next, if  $g(a)>0$, we will consider the image of $f(a)$ under $g$.
	% But, the definition of $f$ bounds this value, \[
	% 	g(f(a))=|f(a)-f(f(a))| < |a - f(a)| = g(a)
	% 	.\]
	%
	% So, we have that $g(f(a))<g(a)$, contracting the fact that $g(a)$ was the minimum of  $g$.
	%
	% Thus,  $g(a)=0$ and the fixed point of  $f$ occurs at  $x=a$.
\end{proof}
\end{document}
