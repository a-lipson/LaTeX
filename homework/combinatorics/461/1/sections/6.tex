\documentclass[../hw1]{subfiles}
\begin{document}
For each nonnegative integer $n$, let $f(n)$ be the number of subsets of $[100]$ which sum to $n$.
For example, $f(6) = 4$ because there are 4 subsets of $[100]$ which sum to 6,
specifically $\{6\}$, $\{1,5\}$, $\{2,4\}$, and $\{1,2,3\}$.
Prove that $f(2000) = f(3050)$.

\emph{Hint:} What is $1 + 2 + \dots + 100$?
\begin{proof}
	Note that the 100th triangle number is $5050=\frac{100(101)}{2}$.

	In order to show that $f(2000) = f(3050)$, we wish to find a bijection between subsets of [100] which sum to 2000 and 3050 respectively.

	Consider the set $S: \{ s \in [100]\ \mid\ \text{sum of elements of }\,s = 2000\} $.
	Then, the complement of $S$ in [100], $S^{c}$, must contain the rest of the numbers, which sum to $5050-2000=3050$.
	So, the sum of the elements of $S^{c}$ is 3050.

	Similarly, if we start with $S^{c}$ whose sum is 3050, then $(S^{c})^{c}=S$ will have a sum of 2000.

	Hence, we can define a bijection $f: S \longmapsto S^{c}$ by the mapping $s \mapsto [100]\setminus s$.

	This map is well defined because every  $s \in S$ has exactly one complement.

	This map is invertible, $[100]\setminus s \mapsto [100]\setminus ([100] \setminus s) = s  \implies f \circ f = f$.
	So $f$ is its own inverse.

	Thus, $f$ is a bijection.

	Since $f$ is a bijection, then $f(2000) = |S| = |S^{c}| = f(3050)$, which is what we wished to show.
\end{proof}
\end{document}
