\documentclass[../hw7]{subfiles}
\begin{document}
\begin{problem}
\item Let $G$ be a simple graph (not necessarily planar). Let $d$ be the maximum degree of a vertex in $G$.
\begin{enumerate}
	\item Prove that $\chi(G) \le d+1$. \emph{Hint:} Use induction.
	\item Find an infinite family of non-isomorphic graphs for which equality holds in the above inequality.
\end{enumerate}
\end{problem}
\begin{proof}[Proof of (a)]
	We will prove the statement by induction on $d$.

	For the base case, when $d=0$, the all vertices must have degree zero and must therefore be not connected. Hence  $\chi(G)=0+1=1$ color for each vertex.

	Assume, that $\chi(G)=d$ for the maximum vertex degree in $G$ of $d-1$.

	Now, for the maximum degree of $d$, we have by the inductive hypothesis that the chromatic number must be at least $d$.

	Then, the vertex with degree $d$, call it $v$ is adjacent to  $d$ other vertices, each of which can be given one of  $d$ colors.

	But, $v$ must have a different color from the other $d$ vertices;
	hence the graph must be  $(d+1)$-colorable.

	Therefore  $\chi(G)\le d+1$.
\end{proof}
\begin{proof}[Proof of (b)]
	Consider the complete graphs $K_n$.
	The degree of each vertex in  $K_n$ is  $n-1$, so $\chi(K_n)=n$.

	Each $K_n$ is not isomorphic to  $K_{n-1}$ as these graphs have a different number of vertices.

	So, for all $n$, we have that the infinite family of complete graphs has their chromatic number equal to one more than their maximum degree.
\end{proof}
\end{document}
