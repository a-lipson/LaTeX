\documentclass[../hw4.tex]{subfiles}

\begin{document}

\subsection*{39}
Show that the proposition holds.
% Suppose that $f$ is continuous on $[a,b]$ and $f(a)=f(b)$.
% Show that f has at least one critical point in $(a,b)$.

\begin{proposition}
    If $f$ is continuous on $[a,b]$ and $f(a)=f(b)$, then $f$ has at least one critical point in $(a,b)$.
\end{proposition}

\begin{proof}
    We will prove the proposition by two cases, either $f$ is differentiable on $(a,b)$, or $f$ is not differentiable on the interval.

    Recall that a critical number is defined as some $c$ on a differentiable $f$ such that $f'(c)=0$ or $f'(c)$ does not exist.

    If $f$ is differentiable on $(a,b)$, then, since $f(a)=f(b)$, by Rolle's Theorem, there exists a $c \in (a,b)$ such that $f'(c)=0$. In this case, $c$ is a critical point.

    If $f$ is not differentiable on $(a,b)$, then there is at least one $\gamma \in (a,b)$ such that $f'(\gamma)$ does not exist. Therefore $\gamma$ is a critical point is this case as well.
\end{proof}


\subsection*{41}
Give an example of a non-constant function that takes on
both its absolute maximum and absolute minimum on every
interval.

Use the Dirichlet function, $f(x) = \begin{cases}
    \alpha, & x \in \mathbb{Q} \\
    \beta, & x \notin \mathbb{Q} \\
\end{cases}$ where $x \in \mathbb{R}$ and $\alpha<\beta$ so that $\alpha$ is the absolute minimum and $\beta$ is the absolute maximum.

For any interval $I \subseteq \mathbb{R}$, there will be some $a,b \in I$ such that $a \in \mathbb{Q}$ and $b \notin \mathbb{Q}$. Therefore, $f$ will attain the value of both $\alpha$ and $\beta$ on any interval, which are the absolute minimum and maximum respectively.

\end{document}