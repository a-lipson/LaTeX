\documentclass[../hw5.tex]{subfiles}

\begin{document}

Prove that a non-constant linear function is uniformly continuous on the real line.
%This should be straight forward. Start with a continuity proof and make sure your delta does not depend on the point you choose.

\begin{proposition}
    The function $f(x)=ax+b, \quad x\neq0$ is uniformly continuous on $x \in \mathbb{R}$. 
\end{proposition}

\begin{proof}
    By the definition on uniform continuity, for every $\epsilon>0$, there is a $\delta>0$ such that, for all $x,y$, $|x-y|<\delta$ implies $\Big|(ax+b)-(ay+b)\Big|<\epsilon$.

    Let $\epsilon>0$ be given.
    
    Define $\delta=\frac{\epsilon}{|a|}$. 
    
    Assume $|x-y|<\delta=\frac{\epsilon}{|a|}$.

    Then,
    \begin{align*}
        \Big|(ax+b)-(ay+b)\Big| &= |ax-ay| \\
        &= |a||x-y| \\
        &< |a|\frac{\epsilon}{|a|} = \epsilon. \\
    \end{align*}

    Since, $\Big|(ax+b)-(ay+b)\Big|<\epsilon$, the proposition holds.
\end{proof}

\end{document}