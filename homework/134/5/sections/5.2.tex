\documentclass[../hw5.tex]{subfiles}

\begin{document}

\subsection*{12}
(a) Given that $P = {x_0,x_1,\ldots,x_n}$ is an arbitrary partition of $[a,b]$, find $L_f(P)$ and $U_f(P)$ for $f(x) = x + 3$.

(b) Evaluate $\int_{a}^{b} f(x) dx$.

Since $f'(x)=1>0$ for all $x \in \mathbb{R}$, then $f$ is strictly increasing.

Then the maximum of $f$ on any part of $P$ is $M_i=x_i+3$ and the minimum is $m_i=x_{i-1}+3$.

Note that $\Delta x_i = x_i - x_{i-1}$ and $\sum_{i=1}^{n} \Delta x_i = \frac{b-a}{n}$.

The upper bound $U_f(P)$ of $f$ on $P$ can be found with $M_i$,
\begin{align*}
    U_f(P) &= \sum_{i=1}^{n} M_i \Delta x_i \\
    &= \sum_{i=1}^{n} \left( x_i+3 \right) \Delta x_i \\
    &= \sum_{i=1}^{n} x_i \Delta x_i + 3\sum_{i=1}^{n} \Delta x_i \\
    &= \sum_{i=1}^{n} x_i \Delta x_i + 3(b-a). \\
\end{align*}

Similarly, the lower bound $L_f(P)$ of $f$ on $P$ can be found with $m_i$,
\begin{align*}
    L_f(P) &= \sum_{i=1}^{n} m_i \Delta x_i \\
    &= \sum_{i=1}^{n} \left( x_{i-1}+3 \right) \Delta x_i \\
    &= \sum_{i=1}^{n} x_{i-1} \Delta x_i + 3(b-a). \\
\end{align*}

So, \[L_f(P)=\sum_{i=1}^{n} x_{i-1} \Delta x_i + 3(b-a), \quad U_f(P)=\sum_{i=1}^{n} x_i \Delta x_i + 3(b-a).\]

For (b), we note that the mean of $L_f(P)$ and $U_f(P)$ is between these two values. So, provided that, $\Delta x >0$, 
\[L_f(P)< \sum_{i=1}^{n} \frac{x_i+x_{i-1}}{2} \Delta x_i + 3(b-a) <U_f(P).\]

Since $\Delta x_i = x_i - x_{i-1}$, the middle term becomes
\begin{align*}
    & \sum_{i=1}^{n} \frac{x_i+x_{i-1}}{2} (x_i-x_{i-1}) + 3(b-a) \\
    &= \frac{1}{2} \sum_{i=1}^{n} x_i^2-x_{i-1}^2+3(b-a).
\end{align*}

We notice that the sum of the alternating terms, after cancellation, ultimately yields $x_n^2-x_0^2$, where $x_n=b,x_0=a$.

So, \[\frac{1}{2} \sum_{i=1}^{n} x_i^2-x_{i-1}^2 = \frac{b^2-a^2}{2}.\]

Then, \[L_f(P)< \frac{b^2-a^2}{2} + 3(b-a) <U_f(P).\]

By uniqueness of the integral,
\[\int_{a}^{b} (x+3) dx = \frac{b^2-a^2}{2} + 3(b-a).\]


\subsection*{32}
Show that the proposition holds.
%Let $P={x_0,x_1,x_2,\ldots,x_{n−1},x_n}$ be a regular partition of the interval $[a,b]$.

\begin{proposition}
    If $f$ is continuous and decreasing on $[a,b]$ and $P$ is a regular partition on $[a,b]$, 
    then $U_f(P)-L_f(P)=[f(a)-f(b)]\Delta x$.
\end{proposition}

Since $f$ is decreasing, we define the minimum $m_i$ and the maximum $M_i$ on the $i^{\text{th}}$ part of $P$ as, \[m_i=f(x_i), \quad M_i = f(x_{i-1}).\]

Since $P$ is regular, $\Delta x$ is constant.

Then, \[L_f(P)=\sum_{i=1}^{n} f(x_i) \Delta x, \quad U_f(P)=\sum_{i=1}^{n} f(x_{i-1}) \Delta x.\]

So, \[U_f(P)-L_f(P) = \sum_{i=1}^{n} \left[ f(x_{i-1}) - f(x_i) \right] \Delta x.\]

Note that the alternating sum, $f(x_0) - f(x_1)+f(x_1)-f(x_2)+\cdots+f(x_{n-1})-f(x_n)$ will reduce to $f(x_0)-f(x_n)$, where $x_0=a$, and $x_n=b$.

So, \[\sum_{i=1}^{n} \left[ f(x_{i-1}) - f(x_i) \right] \Delta x = \left[ f(a) - f(b) \right] \Delta x\]

Therefore, the proposition holds.

\subsection*{25--30}
Assume that $f$ and $g$ are continuous, that $a<b$, and that \[\int_{a}^{b} f(x)dx > \int_{a}^{b} g(x)dx.\]

Which of the statements necessarily holds for all partitions $P$ of $[a,b]$? 
Justify your answer.

Let $I_f = \int_{a}^{b} f(x)dx, \quad  I_g = \int_{a}^{b} g(x)dx$. So $I_g < I_f$.

From the definition of the integral in 5.2.6, we construct the following relationship,
\begin{equation}\label{eqn:one}
    L_g(P)\leq I_g < I_f \leq U_f(P).
\end{equation}

\subsubsection*{25}
$L_g(P)<U_f(P)$

Clearly holds by equation~\ref{eqn:one}.

\subsubsection*{26}
$L_g(P)<L_f(P)$

By counterexample, we will provide an $f$ and $g$ such that $L_g(P)\geq L_f(P)$.

Use the course partition $P$ on the interval $I=[0,c]$ of $\{0,c\}$ such that $\Delta x=c$.

Let $f(x)=x, \quad g(x)=0$.

Since $f$ is increasing on $I$, its lower bound on the coarse partition will occur at the beginning of $P$, at $x=0$.

Since $g$ is constant, its lower and upper bounds will be the same for all $x$ in $P$.

So, $f(0)=0=g(0)$ and therefore $L_f(P)=L_g(P)$.

We also see that, while $f$ has some positively signed area under its curve between the $x$-axis, $g$ has no area under its curve.

So, $I_f > I_g$ still holds.

Therefore, by counterexample, the statement does not always hold.

\subsubsection*{27}
$L_g(P)<\int_{a}^{b} f(x)dx$

Follows from equation~\ref{eqn:one} and the definition of $I_f$.

\subsubsection*{28}
$U_g (P) < U_f(P)$

Let $f(x)=1, \quad g(x)=x$.

Use the course partition $P$ on $[0,1]$ of $\{0,1\}$.

Since $f$ is constant, then $U_f(P)=1$.

Since $g$ is strictly increasing throughout $P$, then $U_g(P)$ will occur on the end of $P$, where $x=1$. So, $U_g(P)=g(1)=1$.

So, \[U_f(P)=U_g(P).\]

By examples (4) and (5) in the textbook, $I_f = 1$ and $I_g = 1/2$.

So the initial condition $I_f > I_g$ is satisfied.

Therefore, the statement does not always hold.

\subsubsection*{29}
$U_f(P) > \int_{a}^{b} g(x)dx$

Also holds by $I_g$ and equation~\ref{eqn:one}.

\subsubsection*{30}
$U_g(P) < \int_{a}^{b} f(x)dx$

Use the coarse partition $P$ on $[0,c]$ of $\{0,c\}$ such that $x_0 = 0$, $x_1 = c$, and $\Delta x=c$.

Define $k>0$ such that $\frac{c}{2} < k \leq c$.

Let $f(x)=k, \quad g(x)=x$.

Then, by example (4) and (5),
\[I_g=\int_{0}^{c} x dx = \frac{c^2}{2}, \quad I_f=\int_{0}^{c} kdx = ck\]

Since $k>\frac{c}{2}$, then $ck>\frac{c^2}{2}$
So, the initial condition $I_f>I_g$ is satisfied.

Since $g$ is increasing, it's upper bound on $P$ occurs at the right endpoint of $P$, at $x=c$. 

So, \[U_g(P)=\sum_{i=0}^{1} g(x_i) \Delta x = \left( g(0)+g(c) \right)c = c^2.\]

Since $0<k\leq c$, then $ck \leq c^2$.

So, $U_g(P) \geq I_f$.

Therefore, the statement does not always hold.

\end{document}