\documentclass[../hw10]{subfiles}

\begin{document}

\subsection*{27}
Slice a sphere along two parallel planes which are a fixed distance apart. Show that the surface area of the band that is obtained depends only on the distance between the planes, not on their locations.

We will consider a sphere of radius $R$.

We will produce the surface area of this sphere via the shells produced by the rotation of horizontal sections of a circle.

Let this circle be parametrized by the equations 
\[x(\theta)=R\cos{\theta}, \quad y(\theta)=R\sin{\theta}.\]

We will define the two parallel planes by the lines $x=a$ and $x=b$ which intersect the circle. 

We note that the angle at at which the planes intersect the circle is given by the triangle formed between the radius of the sphere and the $y$-coordinate of the intersecting plane. 

So, we let the parallel planes be defined by the two angles, $\alpha$ and $\beta$, which form the bounds of the inner region of the sphere.
\begin{align*}
    \sin{\alpha}&=\frac{a}{R}, \quad \alpha=\sin^{-1}{\frac{a}{R}}; \\
    \sin{\beta}&=\frac{b}{R}, \quad \beta=\sin^{-1}{\frac{b}{R}}. \\
\end{align*}

We then consider each shell piece of area to be the circumference of the shell, which $2\pi$ times the radius $x(\theta)=R\cos{\theta}$, times the length of the outside of the shell piece. The length of of the outside $s$ is given by $s(\theta)=R\theta$

So, each piece of area is given by,
\[\Delta A \approx 2\pi R\cos{\theta} \Delta s.\]

We approximate $\Delta A \approx dA = 2\pi R \cos{\theta} ds$ and integrate between the bounds $\alpha$ and $\beta$.
\begin{align*}
    \int_{\alpha}^{\beta} dA &= \int_{\alpha}^{\beta} 2\pi R \cos{\theta} \, ds \\
    &= 2\pi R \int_{\alpha}^{\beta} \cos{\theta} R \, d\theta \\
    &= 2\pi R^2 \int_{\alpha}^{\beta} \cos{\theta}\, d\theta \\
    &= 2\pi R^2 {[\sin{\theta}]}_{\alpha}^{\beta} \\
    &= 2\pi R^2 (\sin{\beta}-\sin{\alpha}) \\
    &= 2\pi R^2 \left( \frac{b}{R} - \frac{a}{R} \right) \\
    &= 2\pi R (b-a). \\
\end{align*}

So, the surface area of the region in between the planes is given by \[2\pi R(b-a).\]

However, this values depend only on the distance between the two planes $(b-a)$, in addition to the radius of the sphere, but not the absolute position of the planes.


\subsection*{Project}
Take a wheel and mark a point on the rim. Call that point $P$. Now roll the wheel, keeping your eyes on $P$. The jumping path described by $P$ is called a \textit{cycloid}. To obtain a mathematical characterization of the cycloid, let the radius of the wheel be $R$ and set the wheel on the $x$-axis so that the point $P$ starts at the origin.

The cycloid can be parametrized by the functions
\[x(\theta)=R(\theta-\sin{\theta}), \quad y(\theta)=R(1-\cos{\theta}).\]

\subsubsection*{2}
\begin{enumerate}[label=\alph*.]
    \item At the end of each arch, the cycloid comes to a cusp. Show that $x'$ and $y'$ are both 0 at the end of each arch.
    \item Show that the area under an arch of the cycloid is three times the area of the rolling circle.
    \item Find the length of an arch of the cycloid.
\end{enumerate}

For (a) we will demonstrate that $x'$ and $y'$ are zero at the each cusp. Each cusp occurs at the end of an arch, where $\theta=2\pi k, k\in\mathbb{Z}$ which satisfies the sine term in $y$ such that $y(\theta)=0$.

We compute the first derivative of the parametrization, 
\[x'=R(1-\cos{\theta}), \quad y'=R\sin{\theta}.\]

We note that at integer multiples of $2\pi$, $1-\cos{\theta}=0$ and $\sin{\theta}=0$. So, it becomes clear that $x'$ and $y'$ are both zero at integer multiples of $2\pi$.

For (b), we know that the area of a circle of radius $R$ is $\pi R^2$. 

We will show that the area under each arch of a cycloid is then $3\pi R^2$.

We will integrate under the first arch, where $\theta\in[0,2\pi]$, recalling that $\cos^2{\theta}=\frac{\cos{2\theta}+1}{2}$.
\begin{align*}
    \int_{0}^{2\pi} y(\theta)x'(\theta)\, d\theta &= \int_{0}^{2\pi} {(R(1-\cos{\theta}))}^2\, d\theta \\
    &= R^2 \int_{0}^{2\pi} 1-2\cos{\theta}+\cos^2{\theta} \, d\theta \\
    &= R^2{\left[ \theta-2\sin{\theta}+\int\cos^2{\theta}\, d\theta \right]}_{0}^{2\pi} \\
    &= R^2{\left[ \theta-2\sin{\theta}+\int\frac{\cos{2\theta}+1}{2}\, d\theta \right]}_{0}^{2\pi} \\
    &= R^2{\left[ \theta-2\sin{\theta}+\frac{1}{2}\left( \frac{\sin{2\theta}}{2} + \theta \right) \right]}_{0}^{2\pi} \\
    &= R^2{\left[ \theta-2\sin{\theta}+\frac{\sin{2\theta}}{4}+\frac{\theta}{2}\right]}_{0}^{2\pi} \\
    &= R^2{\left[ \frac{3\theta}{2}-2\sin{\theta}+\frac{\sin{2\theta}}{4}\right]}_{0}^{2\pi} \\
    &= R^2\left( \frac{3(2\pi)}{2} \right) \\
    &= 3\pi R^2. \\
\end{align*}

So, the area under the cycloid is three times the area of the circle which traces it.

For (c) we determine the arclength directly, employing the half angle identity $\sin{\frac{x}{2}}=\sqrt{\frac{1-\cos{\theta}}{2}}$ for $x\in[0,2\pi]$.
\begin{align*}
    \int_{0}^{2\pi} \sqrt{{[x'(\theta)]}^2+{[y'(\theta)]}^2} \, d\theta &= \int_{0}^{2\pi} \sqrt{{(R(1-\cos{\theta}))}^2+{R\sin{\theta}}^2} \, d\theta\\
    &= R \int_{0}^{2\pi} \sqrt{1-2\cos{\theta}+\cos^2{\theta}+\sin^2{\theta}} \, d\theta \\
    &= R \int_{0}^{2\pi} \sqrt{2-2\cos{\theta}} \, d\theta \\
    &= 2R \int_{0}^{2\pi} \sqrt{\frac{1-\cos{\theta}}{2}} \, d\theta \\
    &= 2R \int_{0}^{2\pi} \sin{\frac{\theta}{2}} \, d\theta \\
    &= -4R {[\cos{\frac{\theta}{2}}]}_{0}^{2\pi} \\
    &= -4R (\cos{\pi}-\cos{0}) \\
    &= -4R(-2) \\
    &= 8R. \\
\end{align*}

So, the arclength of one arch of the cycloid is eight times the radius of the circle that describes it.

\subsubsection*{3}
\begin{enumerate}[label=\alph*.]
    \item Locate the centroid of the region under the first arch of the cycloid.
    \item Find the volume of the solid generate by revolving the region under an arch of the cycloid about the $x$-axis.
    \item Find the volume of the solid generate by revolving the region under an arch of the cycloid about the $y$-axis.
\end{enumerate}

For (a), by symmetry of each arch region, we expect that the $x$-coordinate of the centroid to lie in the middle of that region. This occurs at half of the width of each region, or $\pi R$.

Then, we compute the $y$-coordinate, $\overline{y}$, using the first region between $\theta=0$ and $\theta=2\pi$. We have already reduced the integrand of the arclength integral, noting that for the length $s$, the sum of the pieces, $\int ds = \int 2R\sin{\theta/2}$ for $\theta\in[0,2\pi]$.

So, with the fact that $\cos{2x}=2\cos^2{x}-1$,\footnote{This fact is clearly demonstrated once we take the identity $\cos{2x}=\cos^2{x}-\sin^2{x}$ along with $\sin^2{x}=1-\cos^2{x}$.}
\begin{align*}
    \overline{y}L &= \int_{0}^{2\pi} y(\theta) \sqrt{{[x'(\theta)]}^2+{[y'(\theta)]}^2}\, d\theta \\
    &= \int_{0}^{2\pi} R(1-\cos{\theta})\left( 2R\sin{\frac{\theta}{2}} \right)\, d\theta \\
    &= 2R^2 \left[ \int_{0}^{2\pi} \sin{\frac{\theta}{2}} \, d\theta - \int_{0}^{2\pi} \cos{\theta}\sin{\frac{\theta}{2}} \, d\theta \right] \\
    &= 2R^2 \left[ 2 {\left[ -\cos{u} \right]}_{0}^{\pi} - \int_{0}^{\pi} \cos{2u}\sin{u} \, du \right] \\
    &= 2R^2 \left[ 2(-\cos{\pi}+\cos{0}) - \int_{0}^{\pi} \left( 2\cos^2{u}-1 \right)\sin{u} \, du \right] \\
    &= 2R^2 \left[ 4 + \int_{\cos{0}}^{\cos{\pi}} \left( 2v^2-1 \right)\, dv \right] \\
    &= 2R^2 \left[ 4 + {\left[ \frac{2v^3}{3}-v \right]}_{1}^{-1} \right] \\
    &= 2R^2 \left( 4 + \left( \left( -\frac{2}{3}+1 \right) - \left( \frac{2}{3}-1 \right)\right) \right) \\
    &= 2R^2 \left( 4 + \left( 2 - \frac{4}{3} \right) \right) \\
    &= 2R^2 \left( \frac{16}{3} \right)\\
    &= \frac{32}{3} R^2. \\
\end{align*}

Since $L=8R$, then $\overline{y}=\frac{\frac{32}{3}R^2}{8R}=\frac{4}{3}R$.

For (b) and (c), we will use Pappus theorem of a planar region rotated about an axis. 
\[V=2\pi\overline{R}A,\]
where $\overline{R}$ is the centroid coordinate against the axis of rotation, and $A$ is the area of the region, $3\pi R^3$.



For (b), the rotation about the $x$-axis we use the $\overline{y}$, so $V=2\pi\left( \frac{4}{3}R \right)(3\pi R^2) = \frac{8\pi^2}{3}R^2$

For (c), we use $\overline{x}$ to rotate about the $y$-axis.
So, $V=2\pi(\pi R)(3 \pi R^2)=6\pi^3 R^3$.


\end{document}