\documentclass[../hw11]{subfiles}

\begin{document}

\subsection*{23}
When an object of mass m moves through air or a viscous medium, it is acted on by a frictional force that acts in the direction opposite to its motion. 
This frictional force depends on the velocity of the object and (within close approximation) is given by 
\[F(v) = -\alpha v - \beta v^2,\] 
where $\alpha$ and $\beta$ are positive constants. 

\begin{enumerate}[label= (\alph*)]
    \item From Newton's second law, $F = ma$, we have 
    \[m \frac{dv}{dv} = -\alpha v- \beta v^2.\]
    Solve this differential equation to find $v(t)$.
    \item Find $v$ if the object has initial velocity $v(0)=v_0$.
    \item What happens to $v(t)$ as $t\to\infty$?
\end{enumerate}

For (a), we begin by solving the %first order nonlinear nonhomogeneous differential equation
differential equation.

First, we will compute the partial fraction decomposition of the structure $\frac{1}{x(1+ax)}$,
\begin{align*}
    \frac{1}{x(1+ax)}&=\frac{A}{x}+\frac{B}{1+ax} \\
    1&=A(1+ax)+B(x) \\
    x=0 &\implies A=1 \\
    x=-\frac{1}{a} &\implies B=-a \\
    \frac{1}{x(1+ax)}&=\frac{1}{x}-\frac{a}{1+ax}.
\end{align*} 

We will also isolate $x$ in $A=\frac{x}{1+ax}$,
\begin{align*}
    \frac{x}{1+ax}&=A \\
    x&=A(1+ax) \\
    x&=A+axA \\
    x-axA&=A \\
    x(1-aA)&=A \\
    x&=\frac{A}{1-aA}. \\
\end{align*}

These two algebraic reductions will aid us in solving the differential equation.

We proceed using the partial fraction decomposition for with the substitution $a=\frac{\beta}{\alpha}$ and the simplification where $A=ce^{-\frac{\alpha}{m}t}$.
\begin{align*}
    m\frac{dv}{dt}&=-\alpha-\beta v^2\\
    m\frac{dv}{dt}&=-\alpha\cdot v\left( 1+\frac{\beta}{\alpha}v\right)\\
    \int \frac{dv}{v\left( 1+\frac{\beta}{\alpha}v \right)} &= \int \frac{-\alpha}{m}\,dt \\
    \int \frac{1}{v}+\frac{-\frac{\beta}{\alpha}}{1+\frac{\beta}{\alpha}v}\, dv &= -\frac{\alpha}{m}\int dt \\
    \ln{|v|}-\ln{\bigg|1+\frac{\beta}{\alpha}v\bigg|}&=c_0-\frac{\alpha}{m}t \\
    \ln{\Bigg|\frac{v}{1+\frac{\beta}{\alpha}v}\Bigg|}&=c_0-\frac{\alpha}{m}t \\
    \Bigg|\frac{v}{1+\frac{\beta}{\alpha}v}\Bigg|&=e^{c_0-\frac{\alpha}{m}t} \\
    \frac{v}{1+\frac{\beta}{\alpha}v}&=ce^{-\frac{\alpha}{m}t} \\
    v&=\frac{ce^{-\frac{\alpha}{m}t}}{1-\frac{\beta}{\alpha}ce^{-\frac{\alpha}{m}t}} \\
    &=\frac{1}{ce^{\frac{\alpha}{m}t}-\frac{\beta}{\alpha}} \\
    v(t)&=\frac{\alpha}{ce^{\frac{\alpha}{m}t}-\beta}. \\
\end{align*}

So, the solution to the differential equation is \[v(t)=\frac{\alpha}{ce^{\frac{\alpha}{m}t}-\beta}.\]

For (b), we consider the initial condition $v(0)=v_0$.
\begin{align*}
    v(0)&=\frac{\alpha}{ce^{\frac{\alpha}{m}(0)}-\beta} \\
    v_0&=\frac{\alpha}{c-\beta} \\
    c-\beta&=\frac{\alpha}{v_0} \\
    c = \frac{\alpha}{v_0}+\beta. \\
\end{align*}

Thus the solution becomes,
\begin{align*}
    v(t)&=\frac{\alpha}{\left( \frac{a}{v_0}+\beta \right)e^{\frac{\alpha}{m}t}-\beta} \\
    &=\frac{\alpha v_0}{(\alpha+\beta v_0)e^{\frac{\alpha}{m}t}-\beta v_0}. \\
\end{align*}

For (c), we consider the terminal behavior of $v(t)$ by taking the limit as $t$ exceeds any number. We will show that,
\[\lim\limits_{t\to\infty}v(t)=0.\]

We will consider the limit,
\[\lim\limits_{t\to\infty}\frac{\alpha v_0}{(\alpha+\beta v_0)e^{\frac{\alpha}{m}t}-\beta v_0}.\]

We notice that the mass of an object $m$, and the given constant $\alpha$ are both positive. So, $\frac{\alpha}{m}$, the coefficient of the exponent term in the denominator is positive. Thus, as $t$ grows without bound, $e^{\frac{\alpha}{m}t}$ will grow as well. As the numerator is constant, an increasing denominator suggest that $v$ will decrease to zero as $t$ grows large.

Thus the terminal behavior of $v(t)$ is that it tends to zero; the object will come to a stop in the viscous medium.


\subsection*{24}
A descending parachutist is acted on by two forces: a constant downward force mg and the upward force of air resistance, which (within close approximation) is of the form $-\beta v^2$ where $\beta$ is a positive constant. We will take the downward direction as positive.

\begin{enumerate}[label= (\alph*)]
    \item Express $t$ in terms of the velocity $v$, the initial velocity $v_0$, and the \textit{terminal velocity} $v_c=\sqrt{\frac{mg}{\beta}}$.
    \item Express $v$ as a function of $t$.
    \item Express the acceleration $a$ as a function of $t$. Verify that the acceleration never changes sign and in time tends to zero.
    \item Show that, in time, $v$ tends to $v_c$.
\end{enumerate}

First we set up the equation to model the force on the parachutist,
\[F(v)=mg-\beta v^2.\]
 
For (a), we note that the force $F(v)$ is the mass $m$ times the acceleration, which is the change in velocity $\frac{dv}{dt}$.

So, we begin by setting up the differential equation and isolating $\int dt = t$.
\begin{align*}
    m\frac{dv}{dt}&=mg-\beta v^2 \\
    \frac{dv}{dt}&=g-\frac{\beta}{m} v^2 \\
    \int \frac{dv}{g-\frac{\beta}{m} v^2} &= \int dt \\
    \frac{1}{g} \int \frac{dv}{1-\frac{\beta}{mg} v^2} &= t. \\
\end{align*}
We notice that $\frac{\beta}{mg}=\frac{1}{{v_c}^2}$. So, the equation becomes,
\begin{align*}
    t &= \frac{1}{g}\int \frac{dv}{1-\frac{v^2}{{v_c}^2}} \\
    &= \frac{{v_c^2}}{g} \int \frac{dv}{{v_c}^2-v^2}.
\end{align*}

Thus, we have expressed $t$ in terms of $v$ and $v_c$. We note that the initial velocity $v_0$ will arise as the constant resulting from the evaluation of the integral $\int dv$.

For (b), we first reduce the partial fraction of a difference of squares to later simplify the computations,
\begin{align*}
    \frac{1}{a^2-x^2}&=\frac{1}{(a+x)(a-x)} \\
    &=\frac{A}{a+x}+\frac{B}{a-x} \\
    1&=A(a-x)+B(a+x) \\
    x=-a &\implies A=\frac{1}{2a} \\
    x=a &\implies B=\frac{1}{2a} \\
    \frac{1}{a^2-x^2}&=\frac{1}{2a}\left( \frac{1}{a+x}+\frac{1}{a-x} \right) 
\end{align*}

We will also perform an algebraic simplification of $\frac{a+x}{a-x}=A$.
\begin{align*}
    \frac{a+x}{a-x}&=A\\
    a+x&=A(a-x)\\
    a+x&=aA-Ax\\
    Ax+x&=aA-a \\
    x(A+1)&=a(A-1) \\
    x=a\frac{A+1}{A-1}. \\
\end{align*}

We will use the equation given in (a) and employ the partial fraction decomposition and simplification above where $a=v_c$, $x=v$, and $A=ce^{\frac{2g}{v_c}t}$.
\begin{align*}
    \frac{{v_c^2}}{g} \int \frac{dv}{{v_c}^2-v^2} &= t \\
    \int \frac{dv}{{v_c}^2-v^2} &= \frac{g}{{v_c}^2}t \\
    \frac{1}{2v_c} \int \left( \frac{1}{v_c+v}+\frac{1}{v_c-v} \right) \,dv &= \frac{g}{{v_c}^2}t \\
    \ln{|v_c+v|}-\ln{|v_c-v|} + c_0&= \frac{2g}{v_c}t \\
    \ln{\bigg|\frac{v_c+v}{v_c-v}\bigg|} &= c_1 + \frac{2g}{v_c}t\\
    \bigg|\frac{v_c+v}{v_c-v}\bigg| &= e^{c_1} e^{\frac{2g}{v_c}t} \\
    \frac{v_c+v}{v_c-v} &= c_2 e^{\frac{2g}{v_c}t} \\
    v(t)&=v_c\cdot\frac{c_2 e^{\frac{2g}{v_c}t}-1}{c_2 e^{\frac{2g}{v_c}t}+1}. \\
\end{align*}


\def\idtanh#1{\frac{e^{2#1}-1}{e^{2#1}+1}}

We can further simplify this solution with the hyperbolic tangent identity,\footnote{From \url{https://en.wikipedia.org/wiki/Hyperbolic_functions}.}
\[\tanh{x}=\frac{e^{2x}-1}{e^{2x}+1}.\]

We rewrite $v(t)$ as
\[v(t)=v_c\cdot\frac{e^{\frac{2g}{v_c}t+c}-1}{e^{\frac{2g}{v_c}t+c}+1}.\]

We then make the substitution $2x=\frac{2g}{v_c}t+c$, such that
\[v(t)=v_c\tanh{\frac{\frac{2g}{v_c}t+c}{2}}=v_c\tanh{\left( \frac{g}{v_c}t+\frac{c}{2} \right)}.\]

% We will also use the sum of arguments identity,\footnote{From \url{https://en.wikipedia.org/wiki/Hyperbolic_functions}.}
% \[\tanh{(x+y)}=\frac{\tanh{x}+\tanh{y}}{1+\tanh{x}\tanh{y}}.\]

% We will show that \[\tanh{\left( b+\frac{c}{2} \right)}=\frac{e^{2b+c}-1}{e^{2b+c}+1}.\]

% We begin with the given sum of arguments identity. We will simplify the numerator first, 
% \begin{align*}
%     \tanh{x}+\tanh{y} &= \idtanh{x}+\idtanh{y} \\
%     &= \frac{\left(e^{2x}-1\right)\left(e^{2y}+1\right)+\left(e^{2y}-1\right)\left(e^{2x}+1\right)}{\left( e^{2x}+1 \right)\left( e^{2y}+1 \right)} \\
%     &= \frac{\left( e^{2x+2y}+e^{2x}-e^{2y}-1 \right)+\left( e^{2x+2y}-e^{2x}+e^{2y}-1 \right)}{e^{2x+2y}+e^{2x}+e^{2y}+1} \\
%     &= \frac{2e^{2x+2y}-2}{e^{2x+2y}+e^{2x}+e^{2y}+1}. \\
% \end{align*}

% Then, we continue with the denominator, 
% \begin{align*}
%     1+\tanh{x}\tanh{y}&=1+\idtanh{x}\cdot\idtanh{y}\\
%     &=1+\frac{e^{2x+2y}-e^{2x}-e^{2y}+1}{e^{2x+2y}+e^{2x}+e^{2y}+1}. \\
% \end{align*}

% We notice that the numerator and denominator both share the common factor of the reciprocal of $e^{2x+2y}+e^{2x}+e^{2y}+1$; we recombine the terms and simplify, yielding,
% \begin{align*}
%     & \frac{2e^{2x+2y}-2}{e^{2x+2y}+e^{2x}+e^{2y}+1+e^{2x+2y}-e^{2x}-e^{2y}+1} \\
%     =& \frac{2e^{2x+2y}-2}{2e^{2x+2y}+2} \\
%     =& \frac{e^{2(x+y)}-1}{e^{2(x+y)}+1} \\
%     =& \tanh{(x+y)}. \\
% \end{align*}

For (c), we note that acceleration $a=\frac{dv}{dt}$. We replace $v(t)$ in the original differential equation, recalling that $v_c=\sqrt{\frac{mg}{\beta}}$.
\begin{align*}
    a(t)&=\frac{dv}{dt} \\
    &=g-\frac{\beta}{m} v^2 \\
    &=g-\frac{\beta}{m} {\left( v_c\tanh{\left( \frac{g}{v_c}t+\frac{c}{2} \right)} \right)}^2 \\
    &=g-\frac{\beta}{m}\left( \frac{mg}{\beta} \right)\tanh^2{\left( \frac{g}{v_c}t+\frac{c}{2} \right)} \\
    &=g\left( 1- \tanh^2{\left( \frac{g}{v_c}t+\frac{c}{2} \right)}\right) \\
    &=g\sech^2{\left( \frac{g}{v_c}t+\frac{c}{2} \right)}. \\
\end{align*}

We can verify this result by differentiating $v(t)$ as well.
\begin{align*}
    a(t)&=\frac{dv}{dt} \\
    &= \frac{d}{dt}\left[v_c\tanh{\left( \frac{g}{v_c}t+\frac{c}{2} \right)}\right] \\
    &= v_c\sech^2{\left(\frac{g}{v_c}t+\frac{c}{2}\right)}\left(\frac{g}{v_c}\right) \\
    &=g\sech^2{\left( \frac{g}{v_c}t+\frac{c}{2} \right)}. \\
\end{align*}

We note that the range of the hyperbolic secant is the interval is bounded below by a horizontal asymptote at zero. Thus, the acceleration $a(t)$, being the product of the gravitational constant $g$ and the square of the hyperbolic secant function, is always positive.

Lastly, given that \[\lim\limits_{x\to\infty} \sech{x} = 0,\]
we see that as $t$ exceeds any given number, then \[\lim\limits_{t\to\infty} a(t) = 0.\]

So, the acceleration tends toward zero in time.

For (d), we observe the limit as the time $t$ exceeds any number of the velocity function $v(t)$,
\[\lim\limits_{t\to\infty} v_c\tanh{\left( \frac{g}{v_c}t+\frac{c}{2} \right)}.\] 

We know that the end behavior of the hyperbolic tangent is that it approaches its vertical bound of one. Thus, 
\[\lim\limits_{t\to\infty} v_c\tanh{\left( \frac{g}{v_c}t+\frac{c}{2} \right)} = v_c.\] 

So, the velocity function approaches the terminal velocity in time.

\subsection*{27}

A rescue package of mass 100 kilograms is dropped from a plane flying at a height of 4000 meters. 
As the object falls, the air resistance is equal to twice its velocity. 
After 10 seconds, the package’s parachute opens and the air resistance is now four times the square of its velocity.

\begin{enumerate}[label= (\alph*)]
    \item What is the velocity of the package the instant the
    parachute opens?
    \item What is the velocity of the package $t$ seconds after the
    parachute opens?
    \item What is the terminal velocity of the package?
\end{enumerate}

We note that it is possible to set up a single differential equation and thereby single solution function for the behavior of the package's fall using the Heaviside step function.

For (a), we begin with the differential equation of the package in free fall to determine the package's velocity at the ten second mark.

The air resistance in free fall is given by the term $-2v$. The term is negated as it opposes the positive downwards direction of motion.

Thus, we model the force being applied to the package as $F=mg-2v$, and arrive at the differential equation for velocity as a function of time as,
\[m\frac{dv}{dt}=mg-2v.\]

We find the general solution to this equation by rearranging the terms to the form $v'+\frac{2}{m}v=g$ and then using the method of an integrating factor as in 9.1.2.
\begin{align*}
    v&=e^{-\int\frac{2}{m}\,dt}\left[ \int e^{\int\frac{2}{m}\,dt}g\,dt \right]\\
    v&=e^{-\frac{2}{m}t}\left[ \int g e^{\frac{2}{m}t} \right] \\
    v&=e^{-\frac{2}{m}t}\left( \frac{mg}{2} e^{\frac{2}{m}t} + c \right) \\
    v&=\frac{mg}{2}+ce^{-\frac{2}{m}t}. \\
\end{align*}

We know that the package started with no vertical velocity. Thus, we consider,
\begin{align*}
    v(0)&=0\\
    &=\frac{mg}{2}+ce^{-\frac{2}{m}(0)} \\
    c&=-\frac{mg}{2}. \\
\end{align*}

So, the equation for the velocity of the package becomes,
\[v(t)=\frac{mg}{2}\left( 1-e^{-\frac{2}{m}t} \right).\]

We then solve for the velocity at the time $t=10$ seconds given that the mass $m=100$kg and the gravitational constant $g$ in standard units is 9.8m/s$^2$.
\begin{align*}
    v(10)&=\frac{100\cdot9.8}{2}\left( 1-e^{-\frac{2}{100}(10)} \right)\\
    &= 490\left( 1-e^{-\frac{1}{5}} \right) \\
    &\approx 88.8. \\
\end{align*}

So, the velocity of the package after ten seconds when the parachute opens is about 88.8 meters per seconds.

For (b), we consider the differential equation where the upwards negative air resistance force is $-4v^2$.

We will solve the equation $m\frac{dv}{dt}=mg-4v^2$ using separation of variables, noting that the initial velocity is given by the solution to (a) when the parachute first opens.

So,
\begin{align}
    \frac{dv}{dt}&=g-\frac{4}{g}v^2\\
    \frac{dv}{dv}&=\frac{4}{m}\left( \frac{mg}{4}-v^2 \right).\\
\end{align}

We will let $k^2=\frac{mg}{4}$, so $\frac{dv}{dt}=\frac{4}{m}\left( k^2-v^2 \right)$.

We proceed in a similar manner to the previous problem, following the given partial fraction decomposition and algebraic simplifications,
\begin{align*}
    \int \frac{dv}{k^2-v^2}&=\int \frac{4}{m}\,dt\\
    \frac{1}{2k}\int \left( \frac{1}{k+v} +\frac{1}{k-v}\right)\,dv&=\frac{4}{m}t+c_0\\
    \ln{\bigg|\frac{k+v}{k-v}\bigg|}&=\frac{8k}{m}t+c_1\\
    \Bigg|\frac{k+v}{k-v}\Bigg|&=e^{\frac{8k}{m}t+c_1}\\
    \frac{k+v}{k-v}&=c_2e^{\frac{8k}{m}t}\\
    v&=k\left( \frac{c_2e^{\frac{8k}{m}t}-1}{c_2e^{\frac{8k}{m}t}+1} \right)\\
    v&=\frac{\sqrt{mg}}{2}\left( \frac{c_2e^{4\sqrt{\frac{m}{g}}t}-1}{c_2e^{4\sqrt{\frac{m}{g}}t}+1} \right)\\
\end{align*}

We will then solve for $c_2$ with the initial velocity $v_0$ of the second equation, $v(0)=v_0\approx88.8$ meters per second, recalling the values of the constants $m$ and $g$.
\begin{align*}
    v(0)&=v_0\\
    &=\frac{\sqrt{mg}}{2}\left( \frac{c_2e^{4\sqrt{\frac{m}{g}}(0)}-1}{c_2e^{4\sqrt{\frac{m}{g}}(0)}+1} \right)\\
    v_0&=\frac{\sqrt{mg}}{2}\left( \frac{c_2-1}{c_2+1} \right)\\
    2v_0(c_2+1)&=\sqrt{mg}(c_2-1) \\
    2 c_2 v_0 + 2 v_0 &= c_2\sqrt{mg} - \sqrt{mg} \\
    2 c_2 v_0 - c_2\sqrt{mg} &= - 2 v_0 - \sqrt{mg} \\
    c_2(2v_0-\sqrt{mg})&=-(2v_0+\sqrt{mg}) \\
    c_2&=\frac{2v_0+\sqrt{mg}}{2v_0-\sqrt{mg}} \\
    &\approx\frac{2\cdot88.8+\sqrt{100\cdot9.8}}{2\cdot88.8+\sqrt{100\cdot9.8}} \\
    &\approx\frac{177.9+99.0}{177.9-99} \\
    &\approx\frac{276.9}{78.9} \\
    &\approx3.51. \\
\end{align*}

So,
\begin{align*}
    v(t)&=\frac{\sqrt{mg}}{2}\left( \frac{3.51e^{4\sqrt{\frac{m}{g}}t}-1}{3.51e^{4\sqrt{\frac{m}{g}}t}+1} \right)\\
    &= \frac{\sqrt{100\cdot9.8}}{2}\left( \frac{3.51e^{4\sqrt{\frac{100}{9.8}}t}-1}{3.51e^{4\sqrt{\frac{100}{9.8}}t}+1} \right)\\
    &= 49.5\left( \frac{3.51e^{12.78t}-1}{3.51e^{12.78t}+1} \right). \\
\end{align*}

For (c), we note that the terminal velocity can be found as $t$ gets very large. 

We can express $v(t)$ in terms of the hyperbolic tangent as before, noting that it approaches one as $t$ gets very large. 
\[v(t)=49.5\tanh{\left( \frac{12.78t+\ln{3.51}}{2} \right)}.\]

So, \[\lim\limits_{t\to\infty} v(t) = 49.5.\]

Thus, the terminal velocity of the package is 49.5 meters per second.

\end{document}