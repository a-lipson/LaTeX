\documentclass[../astro_4]{subfiles}
\begin{document}
\begin{problem}
Calculate how far in AU the Moon would have to be from the Sun for it to retain a nitrogen atmosphere.

Hint: set both $v_{\text{gas}}$ equations equal to each other and solve for the missing variable.
Does your result make sense?
Think about velocities and temperatures to explain.
\end{problem}
Since we are trying to determine what distance the moon would have to be to hold a nitrogen atmosphere and temperature relates to distance, then we will fix all other variables and solve for temperature in our equation.

The moon has an escape velocity of $2300$ m/s, so gas must achieve a speed of at least  $\frac{1}{6}2300\approx 383$ m/s to escape\dots

Nitrogen gas has a molecular weight of 28 amu.

Then, solving for a temperature $T$ such that the nitrogen gas will not exceed the escape velocity of the moon for geologic time, we have
\begin{align*}
	157\sqrt{\frac{T}{28}} & = 383                                          \\
	T                      & = {\left( \frac{383\sqrt{28}}{157} \right) }^2 \\
	T                      & \approx 167 \text{ K}
	.\end{align*}

For a temperature of 167 K, the moon would be located at around 6 AU instead of its current 1 AU.

This answer makes sense because the only variable we are changing is the distance, and therefore temperature, of the moon.
Since the moon keeps its same gravity, the only way to increase its atmosphere holding capabilities is by reducing the kinetic energy of such atmospheric gasses.
This is accomplished by lowering their temperature through increasing the distance from the sun.
\end{document}
