% structure labels
\newtheoremstyle{non-italic-style}
{1em}
{.5em}
{\normalfont}
{}
{\bfseries}
{.}
{.5em}
{}

\newtheoremstyle{numbered-theorem}% name of the style to be used
{\topsep}% measure of space to leave above the theorem. E.g.: 3pt
{\topsep}% measure of space to leave below the theorem. E.g.: 3pt
{\normalfont}% name of font to use in the body of the theorem
{0pt}% measure of space to indent
{\bfseries}% name of head font
{.}% punctuation between head and body
{.5em}% space after theorem head; " " = normal interword space
{\thmname{#1}\thmnumber{ #2}\textnormal{\thmnote{ (#3)}}}

% \theoremstyle{non-italic-style}
\theoremstyle{numbered-theorem}

% \newtheorem*{env tag}{Env Display Name}
\newtheorem{lemma}{Lemma}
\newtheorem{proposition}{Proposition}
\newtheorem{theorem}{Theorem}
\newtheorem{definition}{Definition}
\newtheorem{eg}{Example}
\newtheorem{notation}{Notation}
% \newtheorem{previouslyseen}{As previously seen}
\newtheorem{remark}{Remark}
\newtheorem{note}{Note}
\newtheorem{observe}{Observe}
\newtheorem{property}{Property}
\newtheorem{intuition}{Intuition}
\newtheorem{problem}{Problem}

\newcommand{\exercise}[1]{%
	\def\@exercise{#1}%
	\subsection*{Exercise #1}
}

\newcommand{\subexercise}[1]{%
	\subsubsection*{Exercise \@exercise.#1}
}

% lecture sub files
\def\@lecture{}
\newcommand{\lecture}[3]{
	\ifthenelse{\isempty{#3}}{
		\def\@lecture{Lecture #1}
	}{
		\def\@lecture{Lecture #1: #3}
	}
	\subsection*{\@lecture}
	\marginpar{\small\textsf{\mbox{#2}}}
}


% headers 
% NOTE: overridden by page_frame
% \pagestyle{fancy}
% \makeatletter
%
% % L, C, R: left, center, right
% % E, O: even, odd
% % \fancyhead[LE,RO]{\@author}
% \fancyhead[RO,LE]{\@lecture} % right odd,  left even
% \fancyhead[RE,LO]{\rightmark}          % right even, left odd
% \fancyfoot[RO,LE]{\thepage}  % right odd,  left even
% \fancyfoot[RE,LO]{}          % right even, left odd
% \fancyfoot[C]{\leftmark}     % center
%
% \makeatother
