
% imports
\usepackage[english]{babel}
\usepackage[letterpaper,top=2cm,bottom=2cm,left=3cm,right=3cm,marginparwidth=1.75cm]{geometry}
\usepackage{amsmath, amssymb, amsthm, amsfonts, mathtools}
\usepackage{mathrsfs}
\usepackage{comment}
\usepackage{emptypage} % hide page number when page is empty
\usepackage{enumitem}
\usepackage{fancyhdr}
\usepackage{graphicx}
\usepackage{hyperref}
\usepackage{ifthen,ifthenx}
\usepackage[skip=15pt,indent=0pt]{parskip} % don't indent paragraphs, leave some space between them
\usepackage{pgfplots}
\usepackage{tikz}
\usetikzlibrary{calc, intersections, pgfplots.fillbetween, angles, quotes}
\usepackage{subfiles}
% \usepackage{yamlvars}

% config
\pgfplotsset{compat=1.18}
\pgfdeclarelayer{ft}
\pgfdeclarelayer{bg}
\pgfsetlayers{bg,main,ft}

% put x \to \infty below \lim
\let\svlim\lim\def\lim{\svlim\limits}

% structure labels
\newtheoremstyle{non-italic-style}
{1em}
{.5em}
{\normalfont}
{}
{\bfseries}
{.}
{.5em}
{}

\newtheoremstyle{numbered-theorem}% name of the style to be used
{\topsep}% measure of space to leave above the theorem. E.g.: 3pt
{\topsep}% measure of space to leave below the theorem. E.g.: 3pt
{\normalfont}% name of font to use in the body of the theorem
{0pt}% measure of space to indent
{\bfseries}% name of head font
{.}% punctuation between head and body
{.5em}% space after theorem head; " " = normal interword space
{\thmname{#1}\thmnumber{ #2}\textnormal{\thmnote{ (#3)}}}

\theoremstyle{non-italic-style}
\newtheorem*{lemma}{Lemma}
\newtheorem*{proposition}{Proposition}
\newtheorem*{theorem}{Theorem}
\newtheorem*{definition}{Definition}
\newtheorem*{eg}{Example}
\newtheorem*{notation}{Notation}
\newtheorem*{previouslyseen}{As previously seen}
\newtheorem*{remark}{Remark}
\newtheorem*{note}{Note}
\newtheorem*{observe}{Observe}
\newtheorem*{property}{Property}
\newtheorem*{intuition}{Intuition}
\newtheorem*{problem}{Problem}


% fix spacing
% http://tex.stackexchange.com/questions/22119/how-can-i-change-the-spacing-before-theorems-with-amsthm
\makeatletter
\def\thm@space@setup{%
	\thm@preskip=\parskip \thm@postskip=0pt
}

% lecture sub files
\def\@lecture{}%
\newcommand{\lecture}[3]{
	\ifthenelse{\isempty{#3}}{%
		\def\@lecture{Lecture #1}%
	}{%
		\def\@lecture{Lecture #1: #3}%
	}%
	\subsection*{\@lecture}
	\marginpar{\small\textsf{\mbox{#2}}}
}

\newcommand{\exercise}[1]{%
	\def\@exercise{#1}%
	\subsection*{Exercise #1}
}

\newcommand{\subexercise}[1]{%
	\subsubsection*{Exercise \@exercise.#1}
}

% page frame
% \pagestyle{fancy}
% \fancyhf{}
% \lhead{\today}
% \rhead{}
% \chead{}

\rfoot{\thepage}

% headers 
\pagestyle{fancy}

% L, C, R: left, center, right
% E, O: even, odd

% \fancyhead[LE,RO]{Alexandre Lipson}

% \fancyhead[RO,LE]{\@lecture} % Right odd,  Left even
% \fancyhead[RE,LO]{}          % Right even, Left odd
%
% \fancyfoot[RO,LE]{\thepage}  % Right odd,  Left even
% \fancyfoot[RE,LO]{}          % Right even, Left odd
% \fancyfoot[C]{\leftmark}     % Center

\makeatother


% snippet shortcuts
\newcommand\N{\ensuremath{\mathbb{N}}}
\newcommand\R{\ensuremath{\mathbb{R}}}
\newcommand\Z{\ensuremath{\mathbb{Z}}}
\renewcommand\O{\ensuremath{\emptyset}}
\newcommand\Q{\ensuremath{\mathbb{Q}}}
\newcommand\C{\ensuremath{\mathbb{C}}}


% figure support 
\usepackage{xifthen}
\usepackage{pdfpages}
\usepackage{transparent}
\newcommand{\incfig}[1]{%
	\def\svgwidth{\columnwidth}
	\import{./figures/}{#1.pdf_tex}
}

% http://tex.stackexchange.com/questions/76273/multiple-pdfs-with-page-group-included-in-a-single-page-warning
% \pdfsuppresswarningpagegroup=1

\author{Alexandre Lipson}
